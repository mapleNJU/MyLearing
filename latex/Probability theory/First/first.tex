\documentclass[12pt,a4paper]{ctexart}
\usepackage{ctex}
\usepackage{emptypage} 
\usepackage{fancyhdr}
\usepackage{amsmath,amsfonts,amssymb,mathtools}
\usepackage{graphicx}
\usepackage{mathptmx}
\usepackage{booktabs}
\usepackage[labelfont=bf]{caption}
\usepackage{indentfirst}
\usepackage{caption}
\usepackage{enumitem}
\usepackage[marginal]{footmisc}
\usepackage{subfigure}
\usepackage{fontspec}
\usepackage{geometry}
\usepackage{setspace}
\usepackage{listings}
\usepackage{xcolor}
\usepackage{float}
\newgeometry{left=3cm,top=2.5cm,bottom=2.5cm,right=3cm}
\setmainfont{Times New Roman}
\setCJKmainfont[BoldFont=SimHei,ItalicFont=KaiTi]{SimSun}

\lstset{
	backgroundcolor=\color{green!10!blue!15},
	rulesepcolor= \color{red!40!blue!100},
	breaklines=true,
	breakatwhitespace=false,
	numbers=left, 
	numberstyle= \small,
	keywordstyle= \color{blue},
	commentstyle=\color{gray}, 
	frame=shadowbox
}

\renewcommand{\baselinestretch}{1.5}

\title{\textbf{概率论与数理统计第一次作业}}

\author{
\\
\Large{麻超 \quad 201300066}
\\[6pt]
{ \large \textit{南京大学人工智能学院}}\\[2pt]
}

\date{}
\newcommand{\supercite}[1]{\textsuperscript{\cite{#1}}}

\begin{document}
\maketitle
\setcounter{page}{1}

\section{}
解:频率是随机事件有限次实验下所得到的结果,概率是实现次数接近无限大时所得到的频率。频率是一个可变值,概率是一个固定的准确值。
\par
在一次随机实验之前无法确定该次实验得到什么结果,这被称为随机现象的偶然性,但是经过大量实验,结果的分布往往能够体现一定的规律。以上遍体现了随机现象中的二重性。
\par
若事件A和B不能同时发生,则称它们为互不相容事件。而所有不属于事件A的基本事件的集合称为A的对立事件。对立事件一定是互不相容事件,但反之则不。在集合中,A的对立事件表示为A相对于全集的补集,该补集的任意一个子集都称为A的对立事件。
\section{}
\textbf{解: }
\subsection{}
$(A-AB)\cup B=A\cup B$

$\overline{(\overline{A}\cup B )} =A\cap \overline{B}=A-B$
\subsection{}
$(A\cup B)-C=(A\cup B)\cap \overline{C}=A\cup B$
\section{}
\begin{itemize}
	\item $\overline{A_1}\cap(\bigcap_{k=2}^n A_k)$
	\item $1-\bigcap_{k=1}^n A_k$
	\item $\bigcup_{k=1}^n (\overline{A_k}\cap \bigcap_{j\neq i}A_j)$
	\item $\overline{\bigcup_{k=1}^n (\overline{A_k}\cap \bigcap_{j\neq i}A_j)\cup (\bigcap_{k=1}^n A_k})$
	\item $\bigcup_{i=1,j>i}^n(\bigcap_{k\neq i,k\neq j}^n A_k )$
	\item $\bigcap_{k=1}^n A_k$
\end{itemize}

\section{}
\subsection{}
\textbf{解: }
$\overline{\bigcup_{i=1}^n A_i}=
	\overline{A_1}\cap \overline{\bigcup_{i=2}^n A_i}\\
	=\overline{A_1}\cap \overline{A_2}\cap \overline{\bigcup_{i=3}^n A_i}\\
	=…\\
	=\overline{A_1}\cap \overline{A_2}\cap \overline{A_3}\cap……\cap\overline\\{A_n}\\
	=\bigcap_{i=1}^n \overline{A_i}$
\subsection{}
\textbf{解: }
$\overline{\bigcap_{i=1}^n A_i}=
	\overline{A_1}\cup \overline{\bigcap_{i=2}^n A_i}\\
	=\overline{A_1}\cup \overline{A_2}\cup \overline{\bigcap_{i=3}^n A_i}\\
	=…\\
	=\overline{A_1}\cup \overline{A_2}\cup \overline{A_3}\cup……\cup\overline\\{A_n}\\
	=\bigcup_{i=1}^n \overline{A_i}$
\section{}
\textbf{解: }
\begin{itemize}
	\item $P(\overline{A}B)=2/3\times 1/5=2/15$
	\item $P(\overline{A}\cup\overline{B})=P(\overline{A\cap B})=1-P(AB)=19/20$
	\item $P(A\cup B\cup C)=P(A)+P(B)+P(C)-P(AB)-P(AC)-P(BC)+P(ABC)\\=1/3+1/5+1/6-1/20-1/20-1/60+1/100$
	\item $P(\overline{A}\overline{B}\overline{C})=2/3\times 4/5\times 5/6=4/9$
	\item $P(\overline{A}\overline{B}C)=2/3\times 4/5\times 1/6=4/45$
	\item $P(\overline{A}\overline{B}\cup C)=P(\overline{A}\overline{B})+P(C)-P(\overline{A}\overline{B}C)=2/3\times 4/5+1/6-4/45=11/18$
\end{itemize}
\section{}
\textbf{解: }$\because P(AB)=P(A)+P(B)-P(A\cup B)$

$\therefore $当$A\subseteq B$时,P(AB)最大,为0.6.

当$A\cup B$最大,即$P(A\cup B)=1$时,P(AB)最小,为0.5.
\section{}
\textbf{解: }$\because P(AB)=P(A)+P(B)-P(A\cup B)
	P(\overline{A}\overline{B})=P(\overline{A\cup B})=1-P(A\cup B)\\
	\therefore 1-P(A\cup B)=P(A)+P(B)-P(A\cup B)\\
	\because P(B)=1/4\\
	\therefore P(A)=3/4$
\section{}
\textbf{解: }$\because P(B-A)=P(A\cup B)-P(A)\\
	P(\overline{A}\overline{B})=P(\overline{A\cup B})=1-P(A\cup B)=0.7
	\therefore P(A\cup B)=0.3
	\therefore P(B-A)=0.2$
\section{}
\textbf{证明: } 由数学归纳法:

奠基:当n=2时,$P(A\cup B)=P(A)+P(B)-P(AB)$,成立。

归纳:设当n=k时原式成立,\\		即$P(\bigcup_{i=1}^k A_i)=\sum_{i = 1}^{k}P(A_i)-\sum_{i<j}P(A_iA_j)+…+(-1)^{k-1} P(A_1,A_2,…,A_k)$成立。\\
则当n=k+1时,有:$P(\bigcup_{i=1}^{k+1} A_i)=P(\bigcup_{i=1}^{k} A_i)+P(A_{k+1})-P(\bigcup_{i=1}^{k+1} A_i\cap A_{k+1})\\
	=\sum_{i=1}^{k+1} P(A_i)-\sum_{i<j}P(A_iA_j)+…+(-1)^{k+1}P(A_1,A_2,…,A_k)-P(\bigcup_{i=1}^s (A_i\cap A_{s+1}))\\
	=\sum_{i=1}^{k+1}P(A_i)-\sum_{i<j}P(A_iA_j)+…+(-1)^{k} P(A_1,A_2,…,A_k+1)$\\
故当n=k+1时,仍然成立。由数学归纳法,得证。
\end{document}