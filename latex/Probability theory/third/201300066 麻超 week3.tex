\documentclass[12pt,a4paper]{ctexart}
\usepackage{ctex}
\usepackage{emptypage} 
\usepackage{fancyhdr}
\usepackage{amsmath,amsfonts,amssymb,mathtools}
\usepackage{graphicx}
\usepackage{mathptmx}
\usepackage{booktabs}
\usepackage[labelfont=bf]{caption}
\usepackage{indentfirst}
\usepackage{caption}
\usepackage{enumitem}
\usepackage[marginal]{footmisc}
\usepackage{subfigure}
\usepackage{fontspec}
\usepackage{geometry}
\usepackage{setspace}
\usepackage{listings}
\usepackage{xcolor}
\usepackage{float}
\usepackage{pifont}
\usepackage{algorithm}
\usepackage{algorithmic}
\newgeometry{left=3cm,top=2.5cm,bottom=2.5cm,right=3cm}
\setmainfont{Times New Roman}
\setCJKmainfont[BoldFont=SimHei,ItalicFont=KaiTi]{SimSun}

\lstset{
	backgroundcolor=\color{green!10!blue!15},
	rulesepcolor= \color{red!40!blue!100},
	breaklines=true,
	breakatwhitespace=false,
	numbers=left, 
	numberstyle= \small,
	keywordstyle= \color{blue},
	commentstyle=\color{gray}, 
	frame=shadowbox
}

\renewcommand{\baselinestretch}{1.5}

\title{\textbf{概率论与数理统计第三次作业}}

\author{
\\
\Large{麻超 \quad 201300066}
\\[6pt]
{ \large \textit{南京大学人工智能学院}}\\[2pt]
}

\date{\today}
\newcommand{\supercite}[1]{\textsuperscript{\cite{#1}}}

\begin{document}
\maketitle
\setcounter{page}{1}

\section*{2.1}
两个事件独立是指它们的发生不会相互影响,即A发生与否与B发生与否没有关系,不代表不可以同时发生,其满足公式$P(AB)=P(A)P(B)$.但是两个事件互不相容是指其不能够同时发生,即发生了A就不会发生B,发生了B就不会发生A,满足$P(AB)=0$.
\section*{2.2}
事件ABC相互独立,则有$P(ABC)=P(A)P(B)P(C),P(AB)=P(A)P(B),P(BC)=P(B)P(C),P(AC)=P(A)P(C)$.

由容斥原理,$P(B\cup C)=P(B)+P(C)-P(BC)=P(B)+P(C)-P(B)P(C)$

$\therefore P(A(B\cup C))=P(AB\cup AC)=P(AB)+P(BC)-P(ABC)$

$P(A)P(B\cup C)=P(A)P(B)+P(A)P(C)-P(A)P(B)P(C)=P(AB)+P(BC)-P(ABC)$

$\therefore P(A(B\cup C))=P(A)P(B\cup C)$

\section*{2.3.22}
\subsection*{1}
设第一次及格为事件A,第二次及格为事件B,能取得某种资格为事件S,由题:

$P(A)=p,P(\overline{A})=1-p,P(B|A)=p,P(B|\overline{A})=p/2$

则$P(S)=P(A)+P(\overline{A})P(B|\overline{A})=p+(1-p)p/2=(-p^2+3p)/2$
\subsection*{2}
$P(A|B)=\frac{P(AB)}{P(B)}=\frac{P(B|A)P(A)}{P(B|A)P(A)+P(B|\overline{A})P(\overline{A})}=\frac{p^2}{p^2+\frac{p}{2}(1-p)}=\frac{2p}{p+1}$

\section*{2.3.27}
\subsection*{1}
$P(AB|A)=P(AB)/P(A),P(AB|A\cup B)=P(AB)/P(A\cup B)$

$\because A\subseteq (A\cup B)$

$\therefore P(A)\leq P(A\cup B)$

$\therefore P(AB|A)\geq P(AB|A\cup B)$
\subsection*{2}
$\because P(A|B)=1$

$\therefore P(AB)=P(B),$即$B\subseteq A$

$\therefore \overline{A}\subseteq \overline{B}$

$\therefore P(\overline{A}\overline{B})=P(\overline{A})$

$\therefore P(\overline{B}|\overline{A})=P(\overline{A}\overline{B})/P(\overline{A})=1$
\subsection*{3}
$P(AC)\geq P(BC) (1),P(A\overline{C})\geq P(B\overline{C})(2)$

设全集为U,则对(2)式有$P(A(U-C))\geq P(B(U-C))$

$\therefore P(A)-P(AC)\geq P(B)-P(BC)$

由(1)式,$P(A)-P(B)\geq P(AC)-P(BC)$

$\therefore P(A)\geq P(B)$
\section*{2.3.28}
设第一种花籽发芽为事件A,第二种花籽发芽为事件B
\subsection*{1}
$P1=P(A)P(B)=0.72$
\subsection*{2}
$P2=1-P(\overline{A})P(\overline{B})=1-0.02=0.98$
\subsection*{3}
$P3=P(A)P(\overline{B})+P(\overline{A})P(B)=0.18+0.08=0.26$
\section*{2.3.30}
\subsection*{1}
\subsubsection*{i}
A为掷骰子掷到奇数,B为掷骰子掷到大于3的数,则有$P(A|B)=1/3,P(A)=1/2$
\subsubsection*{ii}
A为掷骰子掷到奇数,B为掷骰子掷到大于0的数,则有$P(A|B)=P(A)=1/2$
\subsubsection*{iii}
A为掷骰子掷到奇数,B为掷骰子掷到小于4的数,则有$P(A|B)=2/3,P(A)=1/2$
\subsection*{2}
由于事件ABC相互独立,故$P(ABC)=P(A)P(B)P(C),P(AB)=P(A)P(B),P(BC)=P(B)P(C),P(AC)=P(A)P(C)$.

\subsubsection*{i}
$P(C(AB))=P(CAB)=P(A)P(B)P(C)=P(AB)P(C)$,故C与AB相互独立.
\subsubsection*{ii}
$P(C(A\cup B))=P(AC\cup BC)=P(AC)+P(BC)-P(ABC)=P(A)P(C)+P(B)P(C)-P(A)P(B)P(C)=P(C)[P(A)+P(B)-P(AB)]=P(C)P(A\cup B)$,故C与$A\cup B$相互独立.
\subsection*{3}
$AB\subset A$,所以若P(A)=0,则$0\leq P(AB)\leq P(A)$

由定义,$P(AB)=0=P(B)\times 0=P(A)P(B)$

所以A,B相互独立.
\subsection*{4}
1.若A,B相互独立,则A,$\overline{B}$也相互独立.

所以$P(A|B)=P(A),P(A|\overline{B})=P(A),P(A|B)=P(A|\overline{B})$

2.设$P(A|B)=P(A|\overline{B})$,则$\frac{P(AB)}{P(B)}=\frac{P(A\overline{B})}{P(\overline{B})}$

$\therefore \frac{P(AB)}{P(B)}=\frac{P(AB)+P(A\overline{B})}{P(B)+P(\overline{B})}=1$

$\therefore P(AB)=P(A)P(B)$,即A与B相互独立.
\section*{2.3.31}
\subsection*{1}
必然错.若互不相容,则$P(AB)=0\neq P(A)P(B)$,不满足相互独立.
\subsection*{2}
必然错.理由同上.
\subsection*{3}
必然错.此时$P(A\cup B)=P(A)+P(B)-P(AB)$,故$P(AB)$最小为0.2,不可能为0,不满足互不相容.
\subsection*{4}
可能对.相互独立与其概率无关系.
\section*{2.3.32}
设第i个人报导为假阳性为事件$A_i$,于是$P(A_i)=0.005,P(\overline{A_i})=0.095$

$\therefore P=1-P(\prod _{i=1}^{140} P(A_i))=1-0.095^{140}$
\section*{2.3.33}
$P(A)=P(B)=P(C)=1/2$.

P(AB),P(AC),P(BC),P(ABC)均表示取得1号球,故$P(AB)=P(AC)=P(BC)=P(ABC)=1/4$

即满足$P(AB)=P(A)P(B),P(BC)=P(B)P(C),P(AC)=P(A)P(C)$,但$P(ABC)\neq P(A)P(B)P(C)$
\section*{2.3.37}
\subsection*{1}
$P1=1-4/7\times 7/9=5/9$
\subsection*{2}
$P2=3/7\times 4/9+2/7\times 2/9=16/63$
\subsection*{3}
$P=p2/(p1+p2)=16/35$
\section*{2.3.39}
$P(B|A_1)=0.98^3,P(B|A_2)=0.9^3,P(B|A_1)=0.1^3$

又$P(A_1)=0.8,P(A_2)=0.15,P(A_3)=0.05$.

由贝叶斯公式得

$P(A_1|B)=\frac{P(B|A_1)P(A_1)}{P(B|A_1)P(A_1)+P(B|A_2)P(A_2)+P(B|A_3)P(A_3)}=\frac{0.98^3\times 0.8}{0.98^3\times 0.8+0.9^3\times 0.15+0.1^3\times 0.05}=0.873$

同理,$P(A_2|B)=\frac{0.9^3\times 0.15}{0.98^3\times 0.8+0.9^3\times 0.15+0.1^3\times 0.05}=0.0.127$

$P(A_3|B)=\frac{0.1^3\times 0.05}{0.98^3\times 0.8+0.9^3\times 0.15+0.1^3\times 0.05}=0.0001$.
\end{document}