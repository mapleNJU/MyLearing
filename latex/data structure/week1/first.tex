\documentclass[12pt,a4paper]{article}
\usepackage{ctex}
\usepackage{emptypage} 
\usepackage{fancyhdr}
\usepackage{amsmath,amsfonts,amssymb,mathtools}
\usepackage{graphicx}
\usepackage{mathptmx}
\usepackage{booktabs}
\usepackage{algorithm}
\usepackage{algorithmic}
\usepackage{enumerate}
\usepackage[labelfont=bf]{caption}
\usepackage{indentfirst}
\usepackage{caption}
\usepackage{enumitem}
\usepackage[marginal]{footmisc}
\usepackage{subfigure}
\usepackage{fontspec}
\usepackage{geometry}
\usepackage{setspace}
\usepackage{listings}
\usepackage{xcolor}
\usepackage{float}
\newgeometry{left=3cm,top=2.5cm,bottom=2.5cm,right=3cm}
\setmainfont{Times New Roman}
\setCJKmainfont[BoldFont=SimHei,ItalicFont=KaiTi]{SimSun}

\lstset{
	backgroundcolor=\color{green!10!blue!15},%代码块背景色
	rulesepcolor= \color{red!40!blue!100}, %代码块边框颜色
	breaklines=true,  %代码过长则换行
	breakatwhitespace=false,
	numbers=left, %行号在左侧显示
	numberstyle= \small,%行号字体
	keywordstyle= \color{blue},%关键字颜色
	commentstyle=\color{gray}, %注释颜色
	frame=shadowbox%用方框框住代码块
}

\renewcommand{\baselinestretch}{1.5}%可加可不加,控制行间距

\title{\textbf{Ps1-201300066}}%标题

\author{
\\
\Large{麻超 \quad 201300066}
\\[6pt]
{ \large \textit{南京大学人工智能学院}}\\[2pt]
}

\date{}
\newcommand{\supercite}[1]{\textsuperscript{\cite{#1}}}

\begin{document}
\maketitle
\setcounter{page}{1}

\section{Problem 1}
解:
1.除此之外,还应证明结果与原数组有相同的元素。因为该算法只有交换两数位置这一过程,没有添加或删除元素,故其正确。

2.循环不变量:A[i……n]中最小元素的位置最多为j.\par
证明:第一次循环时显然可得。\par
设j=k,即证明A[i……n] 的最小元素的位置至多为k.则需要比较A[k]与A[k-1].在进行循环时,需要比较A[k]与A[k-1]的大小,易得无论这二者大小如何,最小的元素必然会存在于先前的k-1个元素之中。得证。

3.循环不变量:A[1,……,i]在循环过程中是按照顺序排列的。\par
证明:当i=1时,可知j=2时,A[1,……,n]中A[1]为最小值,自然按顺序排列。\par
设当i=k时成立,即A[1,……,k]为按顺序排列。由b可知A[k,……,n]中A[k]为最小值。\par
则当i=k+1时,同样由b可知,A[k+1,……,n]中A[k+1]为最小值。\par
所以A[1,……,k+1]按顺序排列。\par
当i=n-1时,循环终止,且A[1,……,n]按顺序排列。\par
由上述归纳法可知,A[1,……,i]在循环过程中是按照顺序排列的。这一结论正确。
\section{Problem 2}
1.T(n) =c1+c2(n+2)+c3(n+1)=(c2+c3)n+(c1+2c2+c3) = $\Theta (n)$.

2.循环不变量:在第i次循环后,$y=\sum_{j=i}^n c_jx^{j-i}$,为不变量。

证明:奠基:当i=n时,$y=c_n+x\times 0=c_nx^0\sum_{j=i}^n c_jx^{j-i} $

归纳:设当i=k时原式成立,即$y=\sum_{j=k}^n c_kx^{j-k}$

则当i=k+1时,$y'=c_{k+1}+x\times y=c_{k+1}+x\times \sum_{j=k+1}^n c_jx^{j-k}=\\
	c_{k+1}+\sum_{j=k}^n c_jx^{j-k+1}=\sum_{j=k+1}^n c_{k+1}x^{j-k-1}$,原式成立。

故由数学归纳法,原式成立。
\section{Problem 3}
\begin{enumerate}
	\item $f\in \Theta(g)$
	\item $f\in O(g)$
	\item $f\in O(g)$
	\item $f\in \Theta(g)$
	\item $f\in \Theta(g)$
	\item $f\in \Theta(g)$
	\item $f\in \Omega(g)$
	\item $f\in \Omega(g)$
	\item $f\in \Omega(g)$
	\item $f\in \Omega(g)$
	\item $f\in \Omega(g)$
	\item $f\in O(g)$
	\item $f\in O(g)$
	\item $f\in \Theta(g)$
	\item $f\in \Omega(g)$
	\item $f\in O(g)$
\end{enumerate}

\section{Problem 4}
$1=n^{1/\lg n}\ll \lg(\lg^{*} n)\ll \lg^{*} n=\lg^{*}(\lg n)\ll 2^{\lg^{*} n}\\
	\ll \ln \ln n\ll \sqrt{\lg n}\ll \ln n\ll \lg^2 n\ll 2^{\sqrt{2 \lg n}}\\
	\ll (\sqrt{2}^{\lg n})\ll n=2^{\lg n}\ll n\lg n=\lg(n!)\ll n^2\ll 4^{\lg n}\\
	\ll n^3\ll (\lg n)!\ll (\lg n)^{\lg n}\ll n^{\lg \lg n}\ll (3/2)^n\\
	\ll 2^n\ll n* 2^n\ll e^n\ll n!\ll (n+1)!\ll 2^{2^n}\ll 2^{2^{n+1}}$
\section{Problem 5}
解:数组为[1……n].其长度为n.\par
设两个栈分别为S,T.令S的栈底元素为A[1],T的栈底元素为A[n],依次向数组中央靠拢。由于数组长度为n,故若两堆栈的元素数量之和小于n,不会发生溢出。设S.top=k,T.top=m,k<m.初始情况下S.top=-1,T.top=n.\par
在一般和未到n的情况下,分别有如下操作:\\
1.Push(S,x):
\par\setlength\parindent{2em}if S.top+1!=T.top
\par\setlength\parindent{4em}S.top=S.top+1
\par\setlength\parindent{4em}S[S.top]=x
\par\setlength\parindent{2em}else return "errorsize"\\
2.Push(T,x):
\par\setlength\parindent{2em}if S.top!=T.top
\par\setlength\parindent{4em}T.top=T.top-1
\par\setlength\parindent{4em}T[T.top]=x
\par\setlength\parindent{2em}else return "errorsize"\\
3.Pop(S):
\par\setlength\parindent{2em}if S.empty()=True:
\par\setlength\parindent{4em}error"underflow"
\par\setlength\parindent{2em}else:
\par\setlength\parindent{4em}S.top=S.top-1
\par\setlength\parindent{4em}return S[S.top+1]\\
4.Pop(T):
\par\setlength\parindent{2em}if T.empty()=True:
\par\setlength\parindent{4em}error"underflow"
\par\setlength\parindent{2em}else:
\par\setlength\parindent{4em}T.top=T.top+1
\par\setlength\parindent{2em}return T[T.top+1]\\
5.S.empty():
\par\setlength\parindent{2em}if S.top=-1
\par\setlength\parindent{4em}return True
\par\setlength\parindent{2em}else return false\\
6.T.empty():
\par\setlength\parindent{2em}if T.top=n
\par\setlength\parindent{4em}return True
\par\setlength\parindent{2em}else return false\\
7.S.size():
\par\setlength\parindent{2em}return S.top+1\\
8.T.size:
\par\setlength\parindent{2em}return n-T.top\\
9.search(S,x):
\par\setlength\parindent{2em}for i in 0 to S.top:
\par\setlength\parindent{2em}if A[i]==x:
\par\setlength\parindent{4em}return True
\par\setlength\parindent{2em}return False\\
10.search(T,x):
\par\setlength\parindent{2em}for i from T.top to n-1:
\par\setlength\parindent{2em}if A[i]==x:
\par\setlength\parindent{4em}return True
\par\setlength\parindent{2em}return False
\section{Problem 6}
解:令S,T为两堆栈.

算法:令如下操作:
\begin{itemize}
	\item Enqueue(x):将X放入S中。
	\item Dequeue(x):若T为空堆栈,则从堆栈S中反复取出元素并将其放入堆栈T中,直到堆栈S为空。
\end{itemize}
经过上述操作后,从T中取出元素,并return it.\par
证明:堆栈的特点为后进先出,队列为先进先出。设x,y为两个元素。\par
first step:Enqueue:将x和y先后入队,即放入S中。\par
next step:Dequeue:出队操作,此时T为空,则将S中元素分别放入T中,由于堆栈执行后进先出原则,则y相较于x先进入T中,之后,从T中取出元素,同样由于堆栈的特点,x相较于y会先出,这样便实现了先进先出的功能。\par
利用数学归纳法证明:\par
奠基:当只有一个元素时,自然正确。\par
归纳:设当有n个元素时正确,则当有n+1个元素时,将前n个元素分为一类,第n+1个元素分为第二类,则和上面next step的算法基本步骤一样,不再赘述。得证。\par
故该算法的正确性得到验证。\par
由于只进行了入队和出队操作,故时间复杂度为$\Theta(1)$ time.


\end{document}