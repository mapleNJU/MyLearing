\documentclass[a4paper]{ctexart}
% \usepackage[margin=1.25in]{geometry}
\usepackage[inner=2.0cm,outer=2.0cm,top=2.5cm,bottom=2.5cm]{geometry}
\usepackage{ctex}
\usepackage{color}
\usepackage{graphicx}
\usepackage{amssymb}
\usepackage{amsmath}
\usepackage{amsthm}
\usepackage{bm}
\usepackage{hyperref}
\usepackage{multirow}
\usepackage{mathtools}
\usepackage{enumerate}

\newcommand{\homework}[5]{
    \pagestyle{myheadings}
    \thispagestyle{plain}
    \newpage
    \setcounter{page}{1}
    \noindent
    \begin{center}
    \framebox{
        \vbox{\vspace{2mm}
        \hbox to 6.28in { {\bf Optimization Methods \hfill #2} }
        \vspace{6mm}
        \hbox to 6.28in { {\Large \hfill #1 \hfill} }
        \vspace{6mm}
        \hbox to 6.28in { {\it Instructor: {\rm #3} \hfill Name: {\rm #4}, StudentId: {\rm #5}}}
        \vspace{2mm}}
    }
    \end{center}
    % \markboth{#4 -- #1}{#4 -- #1}
    \vspace*{4mm}
}

\def\R{\mathbb{R}}
\def\dom{\mathrm{dom}}
\DeclareMathOperator*{\argmin}{argmin}

\newenvironment{solution}
{\color{blue} \paragraph{Solution:\\}}
{\newline \qed}

\begin{document}
%==========================Put your name and id here==========================
\homework{Homework 3}{Fall 2021}{Lijun Zhang}{麻超}{201300066}

\paragraph{Notice}
\begin{itemize}
    \item The submission email is: \textbf{zhangzhenyao@lamda.nju.edu.cn}.
    \item Please use the provided \LaTeX{} file as a template. If you are not familiar with \LaTeX{}, you can also use Word to generate a \textbf{PDF} file.
\end{itemize}

\paragraph{Problem 1: One inequality constraint}
~\\
~\\
With $c\neq 0$, express the dual problem of
\begin{align*}
    \min \quad        & c^\top x   \\
    \text{s.t.} \quad & f(x)\leq 0
\end{align*}
in terms of the conjugate $f^\star$.
~\\

\begin{solution}
    当$\lambda=0$时,有$g(\lambda)=\inf c^Tx=-\infty$

    当$\lambda>0$时,有$g(\lambda)=\inf (c^Tx+\lambda f(x))=\lambda \inf ((\frac{c}{\lambda})^Tx+\lambda f(x))=-\lambda f_1^*(-\frac{c}{\lambda} )$

    其dual problem为:
    \begin{align*}
         & \text{minimize } -\lambda f_1^*(-\frac{c}{\lambda} ) \\
         & \text{subject to } \lambda \geq 0
    \end{align*}
\end{solution}

\paragraph{Problem 2: KKT conditions}
~\\
~\\
Consider the problem
\begin{align*}
    \min _{x \in \mathbb{R}^{2}} \quad & x_{1}^{2}+x_{2}^{2}                                      \\
    \text { s.t. } \quad               & \left(x_{1}-1\right)^{2}+\left(x_{2}-1\right)^{2} \leq 2 \\
                                       & \left(x_{1}-1\right)^{2}+\left(x_{2}+1\right)^{2} \leq 2
\end{align*}
where $x=\left[\begin{array}{ll}x_{1} & x_{2}\end{array}\right]^{\top} \in \mathbb{R}^{2}$. \\
\begin{enumerate}[(1)]
    \item Write the Lagrangian for this problem.
    \item Does strong duality hold in this problem?
    \item Write the KKT conditions for this optimization problem.
\end{enumerate}

\begin{solution}
    a)可以得到其Lagrangian为$L(x,\lambda)=x_1^2+x_2^2+\lambda_1((x_1-1)^2+(x_2-1)^2-2)+\lambda_2(x_1-1)^2+(x_2+1)^2-2),$其中$\lambda \in R^2$

    b)首先,由于$\nabla^2x_1^2+x_2^2=\begin{bmatrix} 2 & 0 \\ 0 & 2 \end{bmatrix}$,该矩阵为正定矩阵,则表明目标函数为凸函数.

    其次,$\nabla^2(x_1-1)^2+(x_2-1)^2-2)=\nabla^2(x_1-1)^2+(x_2+1)^2-2)=\begin{bmatrix} 2 & 0 \\ 0 & 2 \end{bmatrix}$,为正定矩阵,表明约束函数为凸函数,则原问题为凸优化问题.

    由于存在$x=\begin{bmatrix} 1 & 0 \end{bmatrix}^T \in relint \text{ }D=R^2$,使得约束不等式$(x_1-1)^2+(x_2-1)^2-2)=(x_1-1)^2+(x_2+1)^2-2)=-1<0$,

    所以满足Slater条件,表明该不等式约束严格成立,即强对偶性成立.

    c)其KKT conditions是\begin{align*}
         & (x_1-1)^2+(x_2-1)^2-2)\leq 0               \\
         & (x_1-1)^2+(x_2+1)^2-2)\leq 0               \\
         & \lambda_1\geq 0                            \\
         & \lambda_2\geq 0                            \\
         & \lambda_1((x_1-1)^2+(x_2-1)^2-2))=0        \\
         & \lambda_2((x_1-1)^2+(x_2+1)^2-2))=0        \\
         & 2x_1+2\lambda_1(x_1-1)+2\lambda_2(x_1-1)=0 \\
         & 2x_2+2\lambda_1(x_2-1)+2\lambda_2(x_2+1)=0
    \end{align*}
\end{solution}


\paragraph{Problem 3: Equality Constrained Least-squares}
~\\
~\\
Consider the equality constrained least-squares problem
\begin{gather*}
    \begin{matrix}
        \text{minimize~~} & \frac{1}{2}\|Ax-b\|_2^2 \\
        \text{subject to} & Gx=h~~                  \\
    \end{matrix}
\end{gather*}
where $A\in\mathbf{R}^{m\times n}$ with $\mathbf{rank}~A=n$, and $G\in\mathbf{R}^{p\times n}$ with $\mathbf{rank}~G=p$.
\begin{enumerate}[(1)]
    \item Derive the Lagrange dual problem with Lagrange multiplier vector $v$.
    \item Derive expressions for the primal solution $x^\star$ and the dual solution $v^\star$.
\end{enumerate}

\begin{solution}
    a)$L(x,v)=\frac{1}{2} \|Ax-b\|_2^2+V^T(Gx-h)=\frac{1}{2} x^TA^TAx+(G^Tv-A^Tb)^Tx-v^Th$

    所以当$x=-(A^TA)^{-1}(G^Tv-A^Tb)$取得最小值.

    也即用拉格朗日乘数向量~$v$~导出的拉格朗日对偶问题为
    \begin{align*}
        g(v)=-2(G^Tv-A^Tb)^T(A^TA)^{-1}(G^Tv-A^Tb)-v^Th
    \end{align*}

    b)其最优条件为$\nabla xL(x,v)=A^T(Ax^*-b)+G^Tv^*=0,Gx^*=h,x^*=-(A^TA)^{-1}(G^Tv-A^Tb)$

    代入$Gx^*=h$可得,$G(A^TA)^{-1}(A^Tb-G^Tv^*)=h$

    $\therefore v^*=G((A^TA)^{-1}(G^T)^{-1}(h-G(A^TA)^{-1}(A^Tb))$
\end{solution}

\paragraph{Problem 4: Negative-entropy Regularization}
~\\
~\\
Please show how to compute
\begin{equation*}
    \argmin_{x\in \Delta^n} \quad b^\top x+c\cdot\sum_{i=1}^n x_i\ln x_i
\end{equation*}
where $\Delta^n=\{x|\sum_{i=1}^n x_i=1,x_i\geq0,i=1,\cdots,n\}$, $b\in\mathbb{R}^n$ and $c\in\mathbb{R}$.

\begin{solution}
    原问题可以转化为求如下优化问题的最优解:
    \begin{align*}
        \text{minimize }b^Tx+c\cdot\sum_{i=1}^{n}x_i\ln x_i,x_i\geq 0 \\
        \text{subject to }I^Tx-1=0
    \end{align*}

    由于凸函数进行保凸运算之后仍为凸函数,则不等式约束函数为凸函数,等式约束函数仿射.所以这是一个凸优化问题,且当$x=[1/n,1/n,...,1/n]^T$时,明显有Slater条件成立,即KKT条件是保证其最优性的充要条件.

    于是引入一个$\lambda \in R^n,v\in R$,得到:

    (1)$-x\leq 0$

    (2)$I^Tx-1=0$

    (3)$\lambda \geq 0$

    (4)$\lambda_i(-x_i)=0,i=1,2,...,n$

    (5)$b_i+c(\ln x_i+1)-\lambda+v=0,i=1,2,...,n$

    可以得到$\lambda$是一个松弛变量,可以消去,于是得到如下条件:

    (1)$x\geq 0$

    (2)$I^Tx-1=0$

    (3)$b_ix_i+cx_i(\ln x_i+1)+vx_i=0,i=1,2,...,n$

    (4)$b_i+c(\ln x_i+1)+v\geq 0,i=1,2,...,n$

    综上可得目标函数$=\sum_{i=1}^{n}-cx_i-vx_i$

    所以可以得到
    $$c+v=
        \begin{cases}
            >0     & \text{  },x_i=e^{(-b_i-v)/c-1} \\
            \leq 0 & \text{  },/
        \end{cases}$$

    $\therefore x_i=e^{(b_i-v)/c-1}$,根据(2)可以得到$\sum_{i=1}^{n}x_i=e^{(b_i-v)/c-1}=1$,所以v有唯一解,满足条件,得到$x_i$的值.
\end{solution}


\paragraph{Problem 5: Support Vector Machines}
~\\
Consider the following optimization problem
\begin{gather*}
    \begin{matrix}
        \text{minimize~~} & \sum_{i=1}^n\max\left(0,1-y_i(w^Tx_i+b)\right)+\frac{\lambda}{2}\|w\|_2^2 \\
    \end{matrix}
\end{gather*}
where $x_i\in\mathbf{R}^{d},y_i\in \mathbf{R},i=1,\cdots,n$ are given, and $w\in \mathbf{R}^d,b\in\mathbf{R}$ are the variables.
\begin{enumerate}[(1)]
    \item Derive an equivalent problem by introducing new variables $u_i,i=1,\cdots,n$ and equality constraints \[u_i=y_i(w^Tx_i+b),i=1,\cdots,n.\]
    \item Derive the Lagrange dual problem of the above equivalent problem.
    \item Give the Karush-Kuhn-Tucker conditions.
\end{enumerate}

\noindent\emph{Hint: Let $\ell(x)=\max(0,1-x)$. Its conjugate function $\ell^\ast(y)=\sup\limits_{x}(yx-\ell(x))=\left\{
\begin{aligned}
     & y, \quad-1\leq y\leq0        \\
     & \infty,\quad\text{otherwise}
\end{aligned}
\right.$}


\end{document}
