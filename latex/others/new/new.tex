\documentclass[12pt,a4paper]{ctexart}
\usepackage{ctex}
\usepackage{emptypage} 
\usepackage{fancyhdr}
\usepackage{amsmath,amsfonts,amssymb,mathtools}
\usepackage{graphicx}
\usepackage{mathptmx}
\usepackage{booktabs}
\usepackage[labelfont=bf]{caption}
\usepackage{indentfirst}
\usepackage{caption}
\usepackage{enumitem}
\usepackage[marginal]{footmisc}
\usepackage{subfigure}
\usepackage{fontspec}
\usepackage{geometry}
\usepackage{setspace}
\usepackage{listings}
\usepackage{xcolor}
\usepackage{float}
\newgeometry{left=3cm,top=2.5cm,bottom=2.5cm,right=3cm}
\setmainfont{Times New Roman}
\setCJKmainfont[BoldFont=SimHei,ItalicFont=KaiTi]{SimSun}

\lstset{
	backgroundcolor=\color{green!10!blue!15},
	rulesepcolor= \color{red!40!blue!100},
	breaklines=true,
	breakatwhitespace=false,
	numbers=left, 
	numberstyle= \small,
	keywordstyle= \color{blue},
	commentstyle=\color{gray}, 
	frame=shadowbox
}

\renewcommand{\baselinestretch}{1.5}

\title{\textbf{史记读书笔记}}

\author{
\\
\Large{麻超 \quad 201300066}
\\[6pt]
{ \large \textit{南京大学人工智能学院}}\\[2pt]
}

\date{}
\newcommand{\supercite}[1]{\textsuperscript{\cite{#1}}}

\begin{document}
\maketitle
\setcounter{page}{1}

鲁迅曾对史记如此评价:史家之绝唱,无韵之离骚。作为我国历史上第一部纪传体史书,史记对后世影响巨大。史记给我们生动描绘了汉武帝时期前的中国历史,虽然部分带有神话的色彩,然而总体价值极高,司马迁也顺理成章地青史留名,成为了广大高中学子作文中的素材。直到两千年后,再次品味司马迁的文字,才能体会到其文字力量之重。

之前我通过一些野史杂谈了解先秦的历史,然而只有在读过史记之后,才可以体会到那个时代的魅力。春秋战国,彼时的中国正在从奴隶制向封建制度转化,各国征战不休,各种思想在这里互相碰撞,诞生了许多有名的著作,为后世留下巨额财富。孔孟儒学,老庄道学……曾想起,自吴起变法之后,列国变法不断。我对秦国发展的历史非常感兴趣,此次也借着这个机会阅读了史记的《秦本纪》,对秦国发展的历史有了一个简单的了解。

秦本纪讲述了秦国祖先在西周建立封地,后因护送有功被封为诸侯国,建立秦国,然中原各国视秦为戎狄之国,认为其过于落后,处处针对,秦国历代在戎族地区扩充土地,另一方面与中原各国来往,通商通婚,逐步成长,吸取中原文化,被称为强国,自献公开始,奋六世之余烈,终于在公元前221年由秦始皇统一了全中国,建立了中国历史上第一个统一的大一统王朝。自此,各个朝代都以大一统为目标,致力于开疆拓土。

在阅读《秦本纪》的过程中,我还粗浅阅读了《秦始皇本纪》《商君列传》《张仪列传》《白起王翦列传》等与之有关的文章,对与之有关的历史有了肤浅的了解。

《秦本纪》记载,秦国祖先为颛顼之女女修,其子大业,大业生子大费,大费曾跟随禹一起治水。治水成功后,舜奖赏了大费,并把一个姚姓美女嫁给大费。后来舜让大费负责驯养鸟兽,并赐姓给他“嬴”——这就是嬴姓的由来;大费有两个玄孙,其中一个名为中衍,中衍的玄孙名为中潏,其生活在西戎地区,守卫着商国的西部边疆。其后辈中有一个名为非子的人,非子在汧水和渭水之间为周孝王养马,后来被周孝王封在秦地——这就是秦国的由来,非子号称“秦嬴”;秦嬴生了秦侯。秦侯在位十年去世。秦侯生公伯。公伯在位三年去世。公伯生秦仲。秦仲即位三年,周厉王无道,有的诸侯背叛了他。西戎族反叛周王朝,灭了犬丘大骆的全族。周宣王登上王位之后,任用秦仲当大夫,讨伐西戎。西戎杀掉了秦仲。秦仲即位为侯王二十三年,死在西戎手里。后其子打败西戎,周宣王于是赏赐其子孙,任命其为西垂大夫。后来周幽王烽火戏诸侯,引来西戎攻击。西戎的犬戎和申侯一起攻打周朝,在郦山下杀死了幽王。秦襄公率兵营救周朝,作战有力,立了战功。周平王为躲避犬戎的骚扰,把都城向东迁到洛邑,襄公带兵护送了周平王。周平王封襄公为诸侯,赐给他岐山以西的土地,自此,秦国正式成为诸侯之一。

后来,穆公即位,这是秦国历史上一位很有名的君主,他在位期间,举贤任能,以五张羊皮买下百里奚的故事至今仍被人津津乐道,后又在百里奚的举荐下任用了蹇叔。他在位期间,将秦的势力范围扩大到黄河流域,收下了河西土地。秦穆公也被称为“春秋五霸”之一(一种说法)。后来进入战国时期,秦国历经数代君主,还经历了一次大的内乱,直到献公以强硬手段夺回君主之位,从他开始,秦国开始扩充力量,奋发图强。献公元年,废除殉葬制度,献公十一年,周朝太史儋会见献公时说:“周固与秦国合而别,别五百岁复合,合十七岁而霸王出”,可谓大预言家。后来在献公带领下,秦国与魏国在石门、少梁等地大战,甚至俘虏了魏将公叔痤。献公二十四年,孝公即位。

秦孝公即位之时,崤山以东有六个较强的国家,秦孝公与齐威王、楚宣王、魏惠王、燕悼侯、韩哀侯、赵成侯并立。然而正如前文所说,山东六国始终以为秦为戎狄之国,不愿与其来往。周室衰微,列国征战不断,而孝公深感秦国国力不足,于是发布了一道求贤令,广招人才。卫鞅听到消息之后,前往秦国,并获得了秦君的赏识,之后卫鞅任左庶长、大良造、商君等职位。商君的变法着重于改革农耕,实行耕战制度。后来,在商君之法的加持下,秦国国力日渐强盛,在与魏国连年的战争中取得优势,夺得安邑,修建咸阳城。并在商君的法令下,秦国正式设县,废除井田制,宣告了郡县制的实现。自商鞅变法之后,秦国逐渐摆脱奴隶制,逐步完成到一个封建王朝的过渡。历史上也提到,自商鞅变法之后,秦国国力日渐强盛,此后与山东六国征战不休,此后的几位君王,励志图强,在100多年以后,实现了大一统。可以说,孝公和商鞅对秦国发展有着决定性的作用,他们制定的法令也是秦国能够在战国时代参与群雄争霸的基础。

孝公死后,其子嬴驷继位,史称为惠文王。他上任后的第一件事情就是处死商鞅,或许是为了平抚国内的保守势力,也有可能是为了报小时候的仇,但我认为惠文王还是一代贤明的君主,他并没有因为商鞅之死而废除国内相关的法令,反而继续发扬光大,且在之后的斗争中平息了国内老氏族等保守势力的反叛。惠文王继位之后,先后任用了犀首公孙衍、张仪等人作为秦相,在与魏国的战争中取得上风,夺回河西之地。与其他各国,如赵国、韩国、楚国等的战争也都逐渐占据了优势地位,甚至夺得楚国六百里土地。秦国在这一阶段,继续学习山东各国的风俗习惯,如举行腊祭。可以说,张仪是那个时代一颗璀璨的明星,其以连横之策,屡次打败山东六国的联合,在秦国与其他各国的外交、战争等方面都取得了卓著的成绩,使秦国真正融入到大国之列,且扩充了秦国的硬实力,如平定义渠,消灭楚国,夺取汉中……为后来的统一战争奠定了基础。

此之后,列国都以秦国为心头之患。惠王死后,其子武王也颇有一番雄心壮志,可惜年少轻狂,在周王室举鼎而亡,但是武王也打下了宜阳城,到达周的都城,向天下宣示了秦的野心。后秦昭襄王即位,开始了超长待机生涯。在秦昭襄王阶段,他先是平定了季君之乱,并逐渐掌握从其母宣太后,其舅魏冉等人手里夺回了权力核心,实现了自己的专权统治,后蔑视周王室权威,自称西帝,后来消灭西周,从事实上基本消灭了周。他在位期间,前期由宣太后和魏冉掌权,后来自己掌权后,任用田文、范雎、白起等人,采取范雎"远交近攻"的战略方针,对赵、韩、魏等三国不断打压,蚕食领地。同时任用的一代良将白起,也是战功赫赫,诸如伊阙之战、鄢郢之战、长平之战……尤其长平之战,在中国历史上十分有名,创下了斩杀40万降卒的记录。后白起死后,在秦昭襄王统治的第51年,秦国率军进攻西周,西周君叩头请罪;五十二年,秦国获得周朝宝器九鼎,宣告周朝基本灭亡;昭襄王一共在位56年,这56年里,秦国国力又有了非常大的提升,对山东各国征战不断,极大地消耗了赵、楚等国的国力,同时联合其他五国伐齐,在昭襄王的手中,山东六国再难有一战之力,基本宣告了秦国统一天下只是时间问题,为秦始皇的统一事业打下了坚实的基础。

昭襄王死后,其子孝文王在位仅仅三天就死了,孝文王的儿子庄襄王即位。庄襄王元年,秦国相国吕不韦诛杀东周君,秦国吞并东周国,自此,西周东周全部灭亡于秦的手中。庄襄王虽仅仅在位三年,但仍然征战不断,战果累累,在他的任内,秦国打通了中原通道,使秦国与齐国接壤。秦庄襄王三年,去世。其子嬴政继位,史称为始皇帝。

秦始皇在位二十六年时,兼并天下,并设置三十六郡。秦始皇去世,其子胡亥即位,即秦二世。秦二世三年(前207年),被赵高所杀,子婴被立为王,子婴在位一个多月,被诸侯杀死,秦朝灭亡。

或许秦国在统一后的历史十分短暂,只有短短十四年,然而是经过了献公以来八位君主的努力,司马迁在《秦本纪》中,如此详细得当地叙写与描绘,着意揭示秦之由弱变强,“文势如阶级”,一层紧一层,可以说为始皇帝最后一统天下,做了令人信服的、合理的铺垫。在秦本纪里,出现了许多栩栩如生的人物,他们共同刻画了那个时代,一步一步地展现了秦国的发家史。在读过《秦本纪》之后,我对先前了解过的秦国历史有了更进一步的了解与理解。从秦的祖先为商汤驾车,直到一千多年后统一全国,这需要的是不断改革发展,不断进取的动力,这部《秦本纪》叫做《秦帝国奋斗史》也不错。或许在鸣条之战(商灭夏之战)中,商君成汤站在秦人(祖先)费昌驾驶的战车上,指挥着商人军队和以周人为代表的联军,向夏人的军队冲去——这是一幅“属于夏人、商人、周人、秦人的历史性同框”。

\end{document}