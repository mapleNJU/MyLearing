\documentclass[12pt,a4paper]{ctexart}
\usepackage{ctex}
\usepackage{emptypage} 
\usepackage{fancyhdr}
\usepackage{amsmath,amsfonts,amssymb,mathtools}
\usepackage{graphicx}
\usepackage{mathptmx}
\usepackage{booktabs}
\usepackage[labelfont=bf]{caption}
\usepackage{indentfirst}
\usepackage{caption}
\usepackage{enumitem}
\usepackage[marginal]{footmisc}
\usepackage{subfigure}
\usepackage{fontspec}
\usepackage{geometry}
\usepackage{setspace}
\usepackage{listings}
\usepackage{xcolor}
\usepackage{float}
\newgeometry{left=3cm,top=2.5cm,bottom=2.5cm,right=3cm}
\setmainfont{Times New Roman}
\setCJKmainfont[BoldFont=SimHei,ItalicFont=KaiTi]{SimSun}

\lstset{
	backgroundcolor=\color{green!10!blue!15},
	rulesepcolor= \color{red!40!blue!100},
	breaklines=true,
	breakatwhitespace=false,
	numbers=left, 
	numberstyle= \small,
	keywordstyle= \color{blue},
	commentstyle=\color{gray}, 
	frame=shadowbox
}
\CTEXsetup[format={\Large\bfseries}]{section}
\renewcommand{\baselinestretch}{1.5}

\title{\textbf{三孩政策解读}}

\author{
\\
\Large{麻超 \quad 201300066}
\\[6pt]
{ \large \textit{南京大学人工智能学院}}\\[2pt]
}

\date{}
\newcommand{\supercite}[1]{\textsuperscript{\cite{#1}}}

\begin{document}
\maketitle
\setcounter{page}{1}
\section{为什么是"三孩"生育政策而不是全面放开}
实施三孩生育政策而不是全面放开,主要有以下三点考虑

1.实施三孩生育政策,体现了国家指导和家庭自主生育相结合的精神。实施“全面两孩”是计划生育,实施“三孩政策”也是计划生育,必须继续坚持计划生育,并赋予计划生育新的内涵。一对夫妻可以生育三个子女,一方面能够满足绝大多数家庭的生育意愿,另一方面也较好地体现了我国人口发展的基本国情。

2.实施三孩生育政策,可以巩固全面建成小康社会成果,达到人口发展与经济社会相适应、资源环境相协调,实现建设社会主义现代化强国目标。

3.实施三孩生育政策,有利于推动实现适度生育水平。假设政策实施后,大部分家庭生2孩、部分家庭生3孩、少部分家庭生1孩,综合起来就会实现适度生育水平的目标。
\section{"三孩"生育政策能不能解决当前中国的人口问题}
"三孩"生育政策不能完全解决中国的人口问题,主要有以下几点考虑:

1.2015年我国实施全面开放二孩政策,出生人口在后一年有所增加,但在后一年便回落了,可以预见的是,中国人口已经不再是简单地开放生育政策可以解决的了,主要是我国经济发展以及居民生活水平,教育水平的提高造就了这种现象,需要更多的助力。

2.生育水平是一个长期指标,随着人均收入水平和教育水平的提升,人口出生率下降符合国际人口出生规律和历史发展规律,中国现在的人口生育率下降或许只是处于适度的低生育水平,在这种情况下,社会会有一定的防范机制。

3.为了缓解人口生育率下降的问题,更多的是配套政策的支持。生育率更多取决于土地供应、房地产政策、教育政策、医疗政策……而不只是开放生育的水平。这些领域如果不改革,而是简单谈促进生育率,固然对于局部家庭有作用,但是作用非常微弱,不可能改变整个趋势。
\end{document}