\documentclass[12pt,a4paper,fontset=none]{ctexart}
\usepackage{ctex}
\usepackage{emptypage} 
\usepackage{geometry}
\newgeometry{left=3cm,top=2.5cm,bottom=2.5cm,right=3cm}
\setmainfont{Times New Roman}
\setCJKmainfont[BoldFont=SimHei,ItalicFont=KaiTi]{SimSun}

\renewcommand{\baselinestretch}{1.5}

\title{\textbf{上学年小结}}

\author{
\\
\Large{南京大学人工智能学院 麻超}
}

\begin{document}
\maketitle
\setcounter{page}{1}

\date{}

\noindent
尊敬的领导和老师:

您好!

我是人工智能学院2020级AI3班的麻超,此次写下的这份上学年小结是对我自己一年以来的各种工作,各种得失作一个小结,对我自己一年来的状况作一个汇报。

去年大约也是在此时,我进入大学,在了解到我的家庭情况后,各位老师给与了我许多照顾,助我缓解家庭经济困难。更要特别感谢瑞华慈善基金会,为我提供了大一至大四持续四年的瑞华启梦助学金。瑞华启梦助学金对我而言是很大的帮助,对每一位身处困境的学生而言都是莫大的帮助,帮助我们这些家庭经济困难学生度过难关,专心钻研学习,争取为国家和社会做出贡献。瑞华筑梦公益社团一直秉承着这样的理念,为每一位家庭经济困难学生带去关怀,带去帮助,将这份助人为乐的意志传承下来,延续精神。

在旧的一年里,我的家庭经济状况略有好转,父母亲依旧在外不辞辛劳地打工,每天早六晚九,十分辛苦。去年我不在家时,母亲还因为多年以来的疾病住了院,去往西安治疗,花费也是颇多。但总的来说,这一年我的家庭状况还是有所好转的,这一切离不开社会和各位老师对我的关心与帮助。

大一一年,我做了许多的志愿工作,体会到了帮助他人的乐趣,在每一次志愿活动中,我都切实得到了提高与信心,体会到用我自己的能力为他人带去帮助的乐趣。另一方面,我在人工智能学院学生会担任了宣传部的部员,在一年的学习经历中,我学习了Ps,Pr等软件的使用,并在科技周上完成了几个项目,在几次的科技周活动中担任摄影摄像工作,与新媒体部对接,为新媒体部推送发放与学生会工作的正常开展提供了助力。在今年7月,AI学生会第二学期末的总结会议上,我被评为宣传部优秀部员,并正式当选为新一届宣传部副部长。目前,我们已经完成了新一届宣传部的招新工作,即将开始进行新部员的培训。未来,我将更多地在学生工作上助力,更好地完成工作。

大一一年我的学习并不是很令自己满意,取得的成绩也并不是特别好,因为平时还没有养成良好的学习习惯,另一方面对大学课程的重视程度不够,轻信依靠抱佛脚就能取得好成绩,事实证明这个结果是错误的。大一一年我共学习了约60学分的课程,除数学分析(二)外,其他课程课程都顺利通过了最后考核,离散数学等课程的考试成绩也比较好,但整体效果并不理想。之后我将积极准备数学分析的补考,并在之后的学习生活中尽力摒弃这种不认真的作风,以严谨求实的态度完成学业。

大学四年里,我将一方面认真完成学业,取得满意的结果,未来选择继续上研究生或者选择进入职场,另一方面,在大学里,发掘自己的兴趣爱好与特长,精进于此,寻找自己喜欢并乐于去做的事情,如学习Ps,Pr就是一种选择。或许还有更多的选择等待我去慢慢探索。在这个框架下,我现有的能力还远远不足,因此希望我能够认真学习专业知识,未来在这一方面发扬自己的力量,做出更深远的影响。

如果有幸能够继续申请这份助学金,我必然会选择珍惜这份补助,缓解家庭经济困难状况,将更多的精力投入到学习生活中去,认真完成学业,绝不辜负各位领导和老师对我的期望。感谢瑞华慈善基金会和瑞华筑梦公益社团带给我这样的机会,我及我的家人由衷地向您表达感谢!向国家,社会,学校,老师等关心我们,支持我们的人表达感激之情!

\end{document}