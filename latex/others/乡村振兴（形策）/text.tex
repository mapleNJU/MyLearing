\documentclass[12pt,a4paper]{ctexart}
\usepackage{ctex}
\usepackage{emptypage} 
\usepackage{fancyhdr}
\usepackage{amsmath,amsfonts,amssymb,mathtools}
\usepackage{graphicx}
\usepackage{mathptmx}
\usepackage{booktabs}
\usepackage[labelfont=bf]{caption}
\usepackage{indentfirst}
\usepackage{caption}
\usepackage{enumitem}
\usepackage[marginal]{footmisc}
\usepackage{subfigure}
\usepackage{fontspec}
\usepackage{geometry}
\usepackage{setspace}
\usepackage{listings}
\usepackage{xcolor}
\usepackage{float}
\newgeometry{left=3cm,top=2.5cm,bottom=2.5cm,right=3cm}
\setmainfont{Times New Roman}
\setCJKmainfont[BoldFont=SimHei,ItalicFont=KaiTi]{SimSun}

\lstset{
	backgroundcolor=\color{green!10!blue!15},
	rulesepcolor= \color{red!40!blue!100},
	breaklines=true,
	breakatwhitespace=false,
	numbers=left, 
	numberstyle= \small,
	keywordstyle= \color{blue},
	commentstyle=\color{gray}, 
	frame=shadowbox
}

\renewcommand{\baselinestretch}{1.5}

\title{\textbf{乡村振兴,“三农”工作振兴的历史性转移}}

\author{
\\
\Large{麻超 \quad 201300066}
\\[6pt]
{ \large \textit{南京大学人工智能学院}}\\[2pt]
}

\date{}
\newcommand{\supercite}[1]{\textsuperscript{\cite{#1}}}

\begin{document}
\maketitle
\setcounter{page}{1}

自中国共产党从1921年7月建立至今已过去100年之多,在这100年里,中国共产党始终是以工农联盟为核心的,代表了广大工农阶级的利益,因此,我党一直坚持为农民谋幸福,谋发展的初心。正因如此,三农工作是我党工作一直以来的重中之重。无论是新中国建立之前我党坚持农村工作,进行土改,或者建国之后1950年代的土地革命行动,又到1978年家庭联产承包责任制的实施,或者进入新世纪以来取消农业税,直到脱贫攻坚,乡村振兴战略,都代表了我党对广大农民群众的关心与支持,也代表我党一直是站在农民群众的战线上的。因此,发展农村经济,实施乡村振兴才显得如此重要,是我国战略上的一步大棋。

另一方面,自2013年实施脱贫攻坚以来至2020年,我国已经基本建成了全面小康,很少有之前那样广大农村与城市格格不入的现象。一方面,在我国实现现代化战略的时候,城镇所占的经济比重与人口数量必然是逐步上升的,这是客观规律,但是另一方面,在现有的客观规律之下,我们应当尽量消除农村与城市之间的发展鸿沟,实现农业现代化,推动乡村振兴,这是必不可少的。尤其在这几年疫情影响之下,国际社会经济发展困难,我国的经济形势已经转变为以内循环为主,在这个背景之下,实施乡村振兴战略更显得重要,以促进我国经济的良好发展。

再者,正如习总书记所说,“绿水青山就是金山银山”,我国现有的经济应大力发展绿色经济,环保经济,在这样的背景下,实施乡村振兴战略,一方面可以促进农村的发展,另一方面可以更好地促进环境保护,建设“产业兴旺、生态宜居、乡风文明、治理有效、生活富裕”的社会主义现代化新农村,是符合绿色经济的发展要求的。

总而言之,实施乡村振兴战略是上承脱贫攻坚伟大历史任务,下接中华民族伟大复兴的重要历史任务,对于实现社会主义伟大复兴,全面建设社会主义现代化国家而言,意义重大而深远。


\end{document}