\documentclass[12pt,a4paper]{ctexart}
\usepackage{ctex}
\usepackage{color}
\usepackage{geometry}
\newgeometry{left=3cm,top=2.5cm,bottom=2.5cm,right=3cm}
\setmainfont{Times New Roman}
\setCJKmainfont[BoldFont=SimHei,ItalicFont=KaiTi]{SimSun}

\CTEXsetup[format={\bfseries}]{section}
\renewcommand{\baselinestretch}{1.5}

\title{\textbf{忆往昔峥嵘岁月,看今朝宏图新秀}}

\author{
\\
\Large{麻超 \quad 201300066}
\\[6pt]
{ \large \textit{南京大学人工智能学院}}\\[2pt]
}
\pagestyle{empty}
\date{}

\begin{document}
\maketitle
\setcounter{page}{1}
2021年11月8日至11日,中国共产党第十九届中央委员会第六次全体会议在北京举行。一年以来,世界百年未有之大变局和新冠肺炎疫情全球大流行交织影响,外部环境更趋复杂严峻,国内新冠肺炎疫情防控和经济社会发展各项任务极为繁重艰巨,在这种情况下,十九届六中全会应时召开。党的十九届六中全会从党和国家事业发展的战略全局出发,深入研究党领导人民进行革命、建设、改革的百年历程,全面总结党从胜利走向胜利的伟大历史进程、为国家和民族建立的伟大历史功绩。十九届六中全会审议通过了《中共中央关于党的百年奋斗重大成就和历史经验的决议》和《关于召开党的第二十次全国代表大会的决议》。

在我看来,这两份文件里前者是总结中国共产党百年以来的伟大成就和历史经验,在建党百年之际,全面开启全面建设社会主义现代化国家新征程,在新时代坚持和发展中国特色社会主义,而后者则是决定党的二十大将于2022年下半年在北京召开。十九届六中全会乃是承上启下之重要会议,承华夏千年之历史,承中国百年之奋斗,启社会主义建设新征程,启在中华民族伟大复兴之梦。
\section{\textbf{\Large{风雨犹祝,山海同欢,是承天地之佑}}}
时间回到1840年,这一年,是中国近代史的开端;这一年,是一个不和平的年份;这一年,英国用枪炮打开了腐朽的清帝国的大门;也正是从这一年开始,历时百年有余,中国人民从古早的封建时代到被列强镇压,这百年不好过,可是中国人民是伟大的,他们用自己的智慧与能力,一次次抵御外辱,从洋务运动,到百日维新,到辛亥革命,到抗日战争,无一不是为了让列强瓜分、吞并中国的美梦落空,让中国人民可以有能力抵御外敌入侵,纵然这之间出了许许多多的插曲,但中国人民始终没有放弃。无论是梁启超先生等坚持的君主立宪,还是孙中山先生主张建立的中华民国,无一不是为了让中国能够富强起来,他们失败了,但精神犹存,这一次次坚持的精神传承下来,以毛泽东同志率领的中国共产党完成了“让中国人民站起来”这一伟大事业。中国共产党是属于人民的政党,经过实践,也只有中国共产党能够带领中国人民成功实现独立,实现富强的伟大梦想,只有马列主义,毛泽东思想才能救中国人民于水火之中。

​自秦始皇嬴政首建大一统之秦帝国以来,中国人民始终秉承和平的发展理念,文景之治、贞观盛世、大宋风华,都是中华太平昌盛的体现。如今,中国更是开启了一个全新的时代,这是属于每一位中国人民的时代,我们上承五千年中华文明,见证今日中国之繁荣昌盛。历时千年岁月,中华太平昌盛依旧,正所谓“风雨犹祝,山海同欢,是承天地之佑”。
\section{\textbf{\Large{星移斗转,沧海桑田,烟火人间依旧}}}
艾青曾言“为什么我的眼里常含泪水,因为我爱这土地爱得深沉”,我相信每一位合格的中国人民都是如此。回望百年之前,当时的中国正处于一片水深火热之中,孙中山先生领导的辛亥革命推翻了清王朝统治,结束了两千多年的君主专制历史,但是没能改变中国的社会性质,中国人民依旧身负“三座大山”。1919年,由于巴黎和会上中国的外交失败,数千名北京学生联合抗议,游行示威,举起了反帝反封建的大旗。在这之后,马克思主义作为一种新的力量出现在历史舞台之上,在李大钊,陈独秀等先辈的努力下,1921年7月中国共产党正式成立。

​中国共产党作为中国最先进的阶级——工人阶级的政党,不仅代表着工人阶级的利益,而且代表着整个中国人民和中华民族的利益。它从一开始就坚持以马克思主义为行动指南,始终把为中国人民谋幸福、为中华民族谋复兴作为初心和使命。中国共产党的成立,是中华民族发展史上开天辟地的大事变,具有伟大而深远的意义。中国共产党的成立,充分展现了开天辟地、敢为人先的首创精神,坚定理想、百折不挠的奋斗精神,立党为公、忠诚为民的奉献精神。这是中国革命精神之源、精神之基、精神之本。

​1921年之后,百年以来,中国共产党历经数次考验,412事变、第五次反围剿失败、长征、张国焘分裂、解放战争,最终在1949年建立了中华人民共和国。建国之后,考验更多,抗美援朝、与苏联的决裂、十年文革、非典、新冠疫情......。不可否认,中国共产党取得的成绩更多,建立了人民当家做主的社会主义国家、改革开放、加入联合国、加入WTO、成为世界经济第二极……,人民的生活水平更是得到了十足的提高,从百年前衣不蔽体食不果腹,到2020年实现全面小康,中国共产党取得了十足的成绩,带领中国人民过上了幸福的生活,正所谓“星移斗转,沧海桑田,烟火人间依旧”。
\section{\textbf{\Large{功名在我,百岁千秋,毋忘秉烛夜游}}}
北宋大家张载有言“为天地立心,为生民立命,为往圣继绝学,为万世开太平”,这是古往今来许多有志向的中华儿女共同的理想。为天地立心,乃心生博爱济众的仁者之心;为生民立命,即广播学识,传道授业,使之安身立命;为往圣继绝学,乃继承先贤之思想,弘扬光大之;为万世开太平,乃所有先贤共同的政治理想。

​在这种思想的影响下,千年来中华民族诞生了无数的英雄人物,他们以中华民族的繁荣昌盛为自己毕生追求的目标,为国为家,鞠躬尽瘁,丰富学识,建功立业。1840年中国开启近代史之后,这样的人物更是数不胜数。林则徐“苟利国家生死以,岂因祸福避趋之”开启虎门销烟运动,以一片赤诚之心维护中华民族的尊严和利益,开眼看世界,让中华民族的儿女们唤起觉醒意识。孙中山先生,为建立一个摆脱封建主义的中国而努力,推翻统治中国几千年的君主专制制度,建立共和政体,推动中华民族思想解放。李大钊先生,坚守共产主义信仰,为中华民族崛起传播新思想,唤醒民众意识。杨靖宇,在国民党政府不抵抗的政策下,毅然前往东北,率领联军阻击日寇,坚持8年之久,在冰天雪地里,靠着草根和棉絮坚持意志,为中华民族献出了自己的生命……

​建国以前,英雄人物不计其数,建国之后,为了中华民族的伟大复兴,也有许多的人延续了这份意志:钱学森,跨越千般万般阻隔,只为回到祖国,为祖国的大业献出自己力量,帮助祖国完成了原子弹和氢弹的研制工作。黄文秀,倾力扶贫,将自己的一生都献给了祖国的扶贫事业,舍身忘我,用生命诠释共产党员的初心和使命。张桂梅,用自己的一生,建立一所华坪女高,点亮乡村女学生的未来之路……还有许许多多的模范,有千千万万的意志等待我们去传承,所有的英雄人物,以自身的风华乃至生命,谱写共产党历史上的一个个传奇故事,正所谓“功名在我,百岁千秋,毋忘秉烛夜游”。
\section{\textbf{\Large{千古诸事,激荡中流,宏图待看新秀}}}
2016年以来,中国的国际形势急转直下,以美国为首的国家对中国采取了敌对的政策,妄图阻碍中国的发展。自2020年以来,受新冠疫情影响,国际形势愈发复杂严峻,国内对抗新冠疫情和发展经济的任务十分艰巨,在这种情况下,却正是数风流人物的今朝时光。

​几年前,因为华为在5G上做出的突出贡献,美国下了一纸禁令,自此华为无法拿到高通的芯片,也难以再产出5G手机。对5G发展做出最大贡献的国内公司却不能使用5G芯片,不能发布5G手机,这何尝不是一种悲哀。这从侧面反映出我国近年经济发展势头特别强盛,但是许多的科技水平没有跟上,尤其在芯片方面,我国的芯片制造被“卡脖子”。因为中国用西方人难以想象的速度发展经济,发展科技,但势必在一些方面有所欠缺,作为新时代的大学生,我们应当秉承着历代先贤的意志,以中华民族的伟大复兴为己任,想想先辈们,在中国羸弱的时候不离不弃,将自己的美好年华,奉献给这个伟大的国家,将自己的一腔热血,献给广袤的土地,我们更应该努力奋斗,为中华民族伟大复兴献出自己的力量。

​今天我学习了十九届六中全会精神,深感历代先贤之不易,感中国共产党百年奋斗之艰辛,感中国特色社会主义制度的优越性。我们大多数人正19、20岁,正是青春岁月,风华正茂时,当有以自身融入中国共产党百年奋斗伟业的觉悟,有毛主席在《沁园春·长沙》里写到的“到中流击水,浪遏飞舟”的豪情,挥洒汗水、不负梦想、以史为鉴、开创未来,埋头苦干、勇毅前行,为实现第二个百年奋斗目标、实现中华民族伟大复兴的中国梦而不懈奋斗,正所谓“千古诸事,激荡中流,宏图待看新秀”!
\end{document}