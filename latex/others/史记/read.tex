\documentclass[12pt,a4paper]{ctexart}
\usepackage{ctex}
\usepackage{emptypage} 
\usepackage{fancyhdr}
\usepackage{amsmath,amsfonts,amssymb,mathtools}
\usepackage{graphicx}
\usepackage{mathptmx}
\usepackage{booktabs}
\usepackage[labelfont=bf]{caption}
\usepackage{indentfirst}
\usepackage{caption}
\usepackage{enumitem}
\usepackage[marginal]{footmisc}
\usepackage{subfigure}
\usepackage{fontspec}
\usepackage{geometry}
\usepackage{setspace}
\usepackage{listings}
\usepackage{xcolor}
\usepackage{float}
\newgeometry{left=3cm,top=2.5cm,bottom=2.5cm,right=3cm}
\setmainfont{Times New Roman}
\setCJKmainfont[BoldFont=SimHei,ItalicFont=KaiTi]{SimSun}

\lstset{
	backgroundcolor=\color{green!10!blue!15},
	rulesepcolor= \color{red!40!blue!100},
	breaklines=true,
	breakatwhitespace=false,
	numbers=left, 
	numberstyle= \small,
	keywordstyle= \color{blue},
	commentstyle=\color{gray}, 
	frame=shadowbox
}

\renewcommand{\baselinestretch}{1.5}

\title{\textbf{史记读书笔记}}

\author{
\\
\Large{麻超 \quad 201300066}
\\[6pt]
{ \large \textit{南京大学人工智能学院}}\\[2pt]
}

\date{}
\newcommand{\supercite}[1]{\textsuperscript{\cite{#1}}}

\begin{document}
\maketitle
\setcounter{page}{1}

鲁迅曾对史记如此评价:史家之绝唱,无韵之离骚。作为我国历史上第一部纪传体史书,史记对后世影响巨大。史记给我们生动描绘了汉武帝时期前的中国历史,虽然部分带有神话的色彩,然而总体价值极高,司马迁也顺理成章地青史留名,成为了广大高中学子作文中的素材。直到两千年后,再次品味司马迁的文字,才能体会到其文字力量之重。史记作为司马迁的集大成之作,全书包括本纪12卷、世家30卷、列传70卷、表10卷、书8卷,共130卷,52万6500余字。其中,书8卷里礼书、乐书、律书、历书、天官书、封禅书、河渠书、平准书。其中平准书是其中非常重要的一本,主要讲述了西汉汉武帝时期平准输送政策的由来。实际上系统介绍了汉武帝以前的富国政策。从中可以看到一个大一统的封建集权政府是如何利用权力,扼杀、限制工商业的发展,以求解决自身财政危机的。其主要措施是改变钱法、卖官爵和卖复徒法、官卖政策(由官卖盐铁发展到平准法的确立)、强制征商等,对于整个封建制度,这是一个探索过程,也给后人留下了深刻教训。

从夏商开始,中国社会处于一个奴隶制社会,这也是几乎所有原始社会发展到一定阶段所形成的社会形态。到周朝时,由于实行分封制,故尤其在东周,列国纷争不断,每一个国家都希望自己能够更加强大,来应对别国的骚扰与侵略,在春秋时,由齐桓公开始,形成了“春秋五霸”的格局,此时的社会整体上还算稳定,虽然列国互有战事,但每一位霸主仍称周王为共主,履行做臣子的职责。自三家分晋,战国伊始,天下大变,由于七个较强的国家占据了整个社会的绝大部分资源,因此大的战事一触即发,然七国军事实力并无大的差距,当此时,更应当注意社会生产力的变化与社会形态的变化。于是就随着铁器的诞生与大范围使用,列国开始变法,从魏国的吴起变法到秦国的商鞅变法,再到赵国的胡服骑射,列国变法都有各自的特点,其性质也大不相同,随着一系列变法的实施,郡县制在列国逐渐推广,且奴隶制逐渐被封建制度所代替等等,同时还有维护平民利益,削弱旧贵族的势力等等,每一次变法就是该国利益的大洗牌(当然有些国家并不是很完全)。其中,最为成功的秦国变法制定了:废除井田制度,实行土地私有制;重农抑商,奖励耕织;统一度量衡;奖励军功,实行二十等爵制;废除“世卿世禄制”,鼓励贵族建立军功;建立严密的户籍制度,制定连坐法;普遍推行县制;制定秦律等等措施。从这里就能看出来一部分对后世的影响,包括在司马迁的平准书里的内容。秦国也伴随着这一套行之有效的政策成功跃升战国里战力最强的国家,并统一全国。秦始皇统一全国后,同样制定了几条经济政策,包括但不限于实行重农抑商政策,承认土地私有化,统一货币,统一度量衡,并将铸币权收归国有。虽然秦始皇提出统一货币货币这一政策时已经是公元前210年,此时的秦始皇已经垂垂老矣,不久之后,秦也在此起彼伏的农民起义中灭亡了,但是统一货币、由国家铸币这样的政策对后世有着极大的影响,此后历朝历代的货币制度都是在此基础上形成的。后来汉朝建立,对原有的经济发展政策继承并改革,创建了一系列新的政策,这也就是平准书所讲的内容。

文章的前几段先是提到汉朝创建初的几位皇帝在经济政策上的措施,汉高祖刘邦创立汉之初,天下百废待兴,国家十分穷困,连皇帝都不能找到四匹颜色一样的马来驾车,百姓没有粮食。刘邦废除了一些秦时期繁重的法令,但却有一些不法分子趁机囤货居奇,造成物价飞涨。高祖至惠帝时期,任然采用重农抑商的策略,只是程度有所不同。文帝景帝之时,卖官鬻爵的现象达到了顶峰,虽然这种现象确实客观上增加了社会的收入,然而也埋下了更大的隐患。

司马迁笔锋一转,提到当今皇上(即武帝)时的状况,社会资产大幅增加,却有着极大的隐患,由于长期的安定,导致官官相护,家族势力横行,乡间有豪强恶党滋生,官府有诸侯公卿相结,于是“物盛而衰,固其变也”。变数随即诞生:全国上下处处爆发战事,南越、闽越一带的战事对江淮的百姓产生了巨大的劳累,司马相如开山凿路扩大巴蜀范围,却使巴蜀百姓疲惫不堪,而对匈奴战事的失利更是严重,直接动摇了北方根基,使天下的百姓都加重了劳役,苦不堪言,同时国家的钱财也无法供应这么大的开销,那么就诞生了让老百姓交钱可以免除刑罚的制度,短期来看,弥补了国库的短暂缺失,然而事实上它打破了官府用人的制度,使得廉洁与耻辱的观念被混淆,朝臣一心谋取利益。后来,卫青击败匈奴,但是向西南开凿道路还未完结,就要向东继续开凿道路,两边作战,百姓的物资消耗巨大,国库日益空虚,于是又颁布了献奴缴羊的人可以升官,免除徭役。再到后来,对匈奴战事继续顺利下去,然而对军队和俘虏的花销却继续增加,这些都是要老百姓来供养的,国库还是空虚,于是继续提出武功爵,即花钱消灾,花钱买官,这样的制度持续下去,官职十分混乱,形同虚设。后面的几段还是一样的道理,继续陈述百姓负担不断加重,而官府应对的策略就是不断地卖官鬻爵,导致百姓民不聊生,官场追逐功利的风气盛行。

第二年,崤山以东爆发水灾,光是安抚灾民就又花了一大笔钱,但还是有官在这里敛财。于是天子与公卿们商议,决定改用新钱重造货币来满足需用,同时打击从事巧取豪夺兼并土地的人。朝廷再次将铸币权收归国有,但是仍然有人顶风作案。朝廷招来新的人才统领此事,使得法律严密,许多官府中的人被免职。但是后来又有新的规定,让商人可以做官,选官的途径更加杂乱了,且商人趁着变更币制之机,囤积了许多货物用以追逐暴利,也有许多人选择去经商,从事农业活动的人越来越少,国家依旧难以行事。对于此事,政府加大了管制力度,天子对卜式赞赏有加,以教化百姓。虽然天子已经颁布缗钱令并且尊崇卜式为天下人的榜样,但百姓终究还是不肯拿出钱财资助朝廷,于是,所谓的告缗案,即告发商人隐瞒财产开始盛行,这个案子还是很有效果的,导致国库逐渐变得富足,中等以上的商人基本上全部破产了。后来继续严加打击私铸钱财,随着钱币的革新,终于渐渐消止了这种行为,只因为代价比不上回报。

后来的情况也逐渐好转,于是由大农令孔仅和桑弘羊提出了平准均输政策,也就是在中央主管国家财政的大司农之下设立均输官,把应由各地输京的物品转运至各处贩卖,从而增加政府收入,抑制商人垄断市场,从而使物价稳定。平准法是汉武帝时期国家平衡物价的政策,在长安和主要城市设立平准官,利用均输官所存物资,根据物价,贵时抛售,贱时收购。实行均输和平准使得京师所掌握的物资大大增加,平抑了市场的物价,打击富商大贾囤积居奇,垄断市场的行为。后来弘羊又请求允许官吏得以缴纳粮食补官,以及罪人纳粮赎罪。文章的最后一段很有意思,卜式说道将桑弘羊下锅煮了天才会下雨,我认为卜式是对桑弘羊政策的不满。

记得初中时学过元朝张养浩的一首元曲,里面有一句“兴,百姓苦;亡,百姓苦”。正如这样,因为秦朝对百姓苛责严重,所以爆发了大规模的农民起义,推翻了秦朝。高祖建立汉以后,国家百废待兴,到文景之时,实行“与民休息,轻徭薄赋”的政策,使得国家国力有所恢复。而到汉武帝之时,由于对匈奴的连年征战,导致国力大幅消耗,然而朝廷的应对策略是什么呢?卖官鬻爵,改变钱法。首先需要承认,这是一个探索的过程,因为之前很少有人能做到这一步,然而其效果并不是很好,直接导致了官府行政混乱,不法分子趁机倒卖,导致社会的贫富差距进一步增大,商人垄断市场,物价飞涨。由于皇帝和各级官员都挥霍无度,百姓真的是民不聊生,汉武帝最后提出的“盐铁官营,平准均输”政策虽然短时间内能解决朝廷财政困难的问题,但在我看来,它并没有解决社会最本质的矛盾,及百姓已经无钱可花。后世的评价中提到由于汉武帝平准均输政策用人不当,一些负责均输、平准的官吏与商人勾结,反而导致物价上涨。在《平准书》的最后,太史公也评价道:“於是外攘夷狄,内兴功业,海内之士力耕不足粮饷,女子纺绩不足衣服。古者尝竭天下之资财以奉其上,犹自以为不足也。无异故云,事势之流,相激使然,曷足怪焉。”准确地分析了汉武帝货币政策的根结在于百姓的生产不能满足消费需求,以及统治者花费过大,激化矛盾。正如新中国一样,作为统治者,应该代表的是广大人民群众的利益,这样才能有人去真心维护你。


\end{document}