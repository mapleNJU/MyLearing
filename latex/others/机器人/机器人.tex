\documentclass[12pt,a4paper]{ctexart}
\usepackage{ctex}
\usepackage{emptypage} 
\usepackage{fancyhdr}
\usepackage{amsmath,amsfonts,amssymb,mathtools}
\usepackage{graphicx}
\usepackage{mathptmx}
\usepackage{booktabs}
\usepackage[labelfont=bf]{caption}
\usepackage{indentfirst}
\usepackage{caption}
\usepackage{enumitem}
\usepackage[marginal]{footmisc}
\usepackage{subfigure}
\usepackage{fontspec}
\usepackage{geometry}
\usepackage{setspace}
\usepackage{listings}
\usepackage{xcolor}
\usepackage{float}
\newgeometry{left=3cm,top=2.5cm,bottom=2.5cm,right=3cm}
\setmainfont{Times New Roman}
\setCJKmainfont[BoldFont=SimHei,ItalicFont=KaiTi]{SimSun}

\lstset{
	backgroundcolor=\color{green!10!blue!15},
	rulesepcolor= \color{red!40!blue!100},
	breaklines=true,
	breakatwhitespace=false,
	numbers=left, 
	numberstyle= \small,
	keywordstyle= \color{blue},
	commentstyle=\color{gray}, 
	frame=shadowbox
}

\renewcommand{\baselinestretch}{1.5}

\title{\textbf{仿生四足机器人的发展综述}}

\author{
\\
\Large{麻超 \quad 201300066}
\\[6pt]
{ \large \textit{南京大学人工智能学院}}\\[2pt]\large \textit{maple@smail.nju.edu.cn}
}

\date{}
\newcommand{\supercite}[1]{\textsuperscript{\cite{#1}}}

\begin{document}
\maketitle
\setcounter{page}{1}
\textbf{摘要} \quad {仿生机器人一直是机器人研究领域内一个非常重要的部分。仿生机器人学是机器人学内一个重要的分支,主要研究以模拟动物的构造、传感系统、计算结构等来构造一个机器人。众所周知,自然界内,四足动物是动物界内数量及其庞大的一个分支,每种动物都有着不同的习性或者运动方式,所以在这些动物内就有许多可以被人们所发掘的用于研究机器人的点。四足机器人属于一种足式机器人,它比轮式和履带式机器人更胜一筹,因为它具有像人类和动物一样在所有地形中探索的潜力。如今,以四足机器人为代表的移动机器人在太空探索、军事应用、工业应用等各个领域都有广泛的应用领域。本文将对多年来四足机器人的发展作一个简单的概述和回顾,并对仿生机器人未来的发展作一个简单的展望} \\

\textbf{关键词}\quad {四足机器人,仿生机器人,步态}
\\[60pt]
\section{引言}
1961 年,贝尔实验室的研究人员在使用 IBM7094 超级计算机研究语音合成时,决定尝试一下合成歌声。John Kelly 和 Carol Lockbaum 负责了人声生成部分,Max Mathews 负责了伴奏生成。他们选择了《Daisy Bell》这首当时已相当知名的爱情歌曲。《Daisy Bell》因此成为了世界上第一首由计算机合成人声演唱的歌曲\supercite{1}。近年来,人工智能技术飞速发展,在声音合成领域取得了非常显著的成绩,也有着非常多的应用,无论是AI配音,以洛天依为代表的虚拟歌手,或者是还原人的声音,成果都非常多。现在流行的“营销号解说”,大多数都是利用了人工智能技术的配音,而从十多年前开始,初音未来、洛天依、乐正绫等一系列虚拟歌手引得人们注意,到现在,他们的技术飞速发展,已经成为了许多晚会不可或缺的项目,其他方面,我们还可用声音合成完成许多过去难以完成的事情,比如在《流浪地球2》电影预告片里就提到了李雪健老师的声音是由AI合成完成的\supercite{2}。
\section{国外的四足机器人发展}
\subsection{1900年代早期的四足机器人}
在1900年代初期,许多科学家和研究人员正致力于研究四足机器人的足机构\supercite{3}。切比雪夫于1870年研制出第一个行走机构,主要是在四杆机构的基础上将旋转运动转化为匀速平移运动,如图1所示。该设备只能在平坦的地形上动态行走,没有独立的腿部运动。后来,这种机制被合并到两台机器 MELWALK 和 DANTE \supercite{4}中。
\begin{figure}[H]
    \centering
    \includegraphics[height=4.5cm]{IMG_01.jpg}
    \caption{切比雪夫的行走机构模型}
\end{figure}
后来,Raibert研制出另一种步行结构,称为“机械马”\supercite{5},如图所示,使用不同类型的连杆和曲柄通过骑手踏板将动力传递给机器。Hutchinson于1940年在英国进行了最早的专门尝试建造具有独立控制腿的机车机器\supercite{6},如图所示。这是一个四足模型,高约60厘米,其八个关节由柔性电缆控制,而柔性电缆又由操作者的两只脚和双手控制。后来,这个概念被通用电气在1960年代用于MOSHER/WALKING TRUCK,如图。
\begin{figure}[H]
    \centering
    \includegraphics[height=4.5cm]{IMG_02.jpg}
    \caption{机械马的结构和实现}
\end{figure}
\subsection{1900年代中期的四足机器人}
美国第一个自主四足机器人于 1960 年代在南加州大学建造。它被命名为“phony pony”\supercite{7}.该机器人的每条腿都有两个相同的旋转关节,具有电动驱动并能够产生多种步态模式像小跑一样,走路包括以缓慢的速度爬行。它还能够保持静止的直立稳定步态,因为每只脚都基于倒T形或骨盆结构,可在正面提供稳定性。

1976年,Hirose和Kato在四足动物的发展中带来了一个重要的里程碑,特别是受到长腿蜘蛛的启发,研制出蜘蛛状四足机器人\supercite{8}。蜘蛛状四足机器人KUMO-1重14kg,长1.5m。每条腿都有一个驱动电机和一个离合器来产生行走运动。1980年代初期,机器人领域发生了一场革命,由Raibert和其在MIT的同事共同完成,他们共同做了一个可以实现类似于单腿袋鼠一样跳跃和奔跑的机器人\supercite{9}。
\subsection{1990年代的四足机器人}
独足平衡和动力学专利原理不仅超越了各种足系统,而且远远超越了任何生物足机器人。它实际上是对四足机器人运动稳定性的控制。Hirose的TITAN 系列\supercite{10}是四足机器人发展中最杰出的成就。首先,他开发了 TITAN-III 四足机器人,其中一个三维受电弓机构结合了一个腿机构,如图3(a)所示。这是世界上第一台攀爬机器人配备智能程序、脚趾触觉传感器和姿态传感器。TITAN-IV 于1986年开发,具有额外的功能,例如它可以通过逐渐增加速度将爬行步行转换为小跑,如图3(b)所示。同样,TITAN-V 和 TITAN-VI 开发用于其他测试目的,如重量、机构、动态行走等。行走四足机器人TITAN-VII\supercite{11}主要用于辅助在陡坡上的施工作业,如移动脚手架,如图3(c)所示。其中,TITAN-VIII\supercite{12}是最受欢迎的工作四足机器人。考虑到重力解耦驱动,它设计有一种新的系留驱动机制(GDA),腿长400m,重量40kg,如图4(a)所示。
\begin{figure}[H]
    \centering
    \includegraphics[height=4.5cm]{IMG_03.jpg}
    \caption{TITAN系列机器人}
\end{figure}
之后的TITAN系列机器人也被用于不同的领域,TITAN-IX用于人道主义排雷任务,如图4(b)所示。TITAN-XI\supercite{13}的性能与 TITAN-VII 相同,但在陡坡上具有钻孔等额外功能,如图4(c)所示。700公斤重的TITAN-XI内置有液压执行器、板载计算机。TITAN-XII\supercite{14}四足机器人可以通过外部计算机和微控制器精确地越过巨大的障碍物以及拥有1.5m/s的速度。
\begin{figure}[H]
    \centering
    \includegraphics[height=4.5cm]{IMG_04.jpg}
    \caption{TITAN系列机器人-2}
\end{figure}
TITAN系列最新的行走四足机器人是TITAN XIII\supercite{15},它是一种伸展型四足机器人,TITAN XIII可以以1.38m/s的速度高速行走,并由电池供电。

1999年,麦吉尔大学的Buchler和Robert设计了两款新型机器人SCOUT-I 和 II\supercite{16}\supercite{17},如图所示。在SCOUT-I中,它的每条腿只有一个驱动自由度,由RC-Servomotor控制臀部运动。SCOUT-II是SCOUT-I的更大版本,其使用工作模型进行爬楼梯模拟以探索动态步态。瑞典皇家理工学院于1998年开发了仿生四足机器人WRAP1\supercite{18},用于研究崎岖地形中的静态和动态运动。WRAP1重约60公斤,共有12个ROF。其由专用的6个控制器局域网(CAN)总线很好地控制的自由度。
\begin{figure}[H]
    \centering
    \includegraphics[height=4.5cm]{IMG_05.jpg}
    \caption{SCOUT-I和SCOUT-II}
\end{figure}
\subsection{21世纪初的四足机器人}
SILO4是西班牙工业自动化研究所(CSIC)于1999年设计的四足步行机器人\supercite{19},如图6所示. 设计师的主要目标是专注于运动生成、地形适应和稳定性。它共有 12 个自由度,并通过电驱动模仿昆虫的腿。在实验上,它用于开发教育机器人和不同研究人员的需求。

成均馆大学机械工程学院为工业公用事业设计了用于墙壁检查的多功能机器人-III(MRWALLSPECT-III),其具有更好的地形适应性,如步入式飞机,使用吸盘爬上带有凸角的墙壁,作为如图7\supercite{20}所示。活动关节由三齿轮直流电机精确驱动。它有两个控制器;一种是使用 Pentium-III 嵌入的,另一种是 Real Time-Linux (RT-Linux)。在MRWALLSPECT-III之后不久,他们又于2007年开发了另一款狗型四足步行机器人,命名为AiDAN-I,随后于2013年开发了AiDAN-III\supercite{21}\supercite{22}。机器人的每个肢体都有3个自由度和一个被动关节,它有一个中央控制器和16个子控制器,每个控制器都嵌入了 CAN配置并由RT-Linux操作。
\begin{figure}[H]
    \centering
    \includegraphics[height=4.5cm]{IMG_07.jpg}
    \caption{Pentium-III}
\end{figure}
\begin{figure}[H]
    \centering
    \includegraphics[height=4.5cm]{IMG_06.jpg}
    \caption{AiDAN-I}
\end{figure}
在控制系统实验室,丰田技术研究所展示了一款名为Robocat-1的电动四足机器人,如图8\supercite{23}所示。在这个机器人中,每条腿有2个总重量为6.85公斤的DOF。这个原型机器人成功地测试了与快速运动控制算法一起使用的小跑循环。
\begin{figure}[H]
    \centering
    \includegraphics[height=4.5cm]{IMG_08.jpg}
    \caption{Robocat-1}
\end{figure}
\section{国内四足机器人发展摘要}
哈尔滨工业大学开发了一种名为MBBOT的四足机器人\supercite{24},其中所有执行器均由外部液压泵提供动力,如图所示。机器人的每条腿都有四个主动关节,一个被动棱柱弹簧安装在脚部和惯性测量单元(IMU) 中的三分量力传感器。车身总重量为100kg,能够在跑步机上以0.83 m/s的速度跑步。
\begin{figure}[H]
    \centering
    \includegraphics[height=4.5cm]{IMG_11.jpg}
    \caption{MBBOT}
\end{figure}
2013年,上海交通大学研制出一种四足机器人,命名为“小象”,因其外形巨大,能够在各种地形下承载重物,如图10(a)所示\supercite{25}。“小象”由4个串并联混合机构支腿组成,由新开发的液压执行器(Hy-Mo)驱动。

北京理工大学介绍了一种基于液压系统的仿生四足动物原型设计,如图10(b)所示\supercite{26}。该机器人共有16个自由度,由汽油发动机提供动力。机器人可以平稳地进行向后、向前和转弯运动。机器人总重量120公斤,最大前进方向小跑速度3公里/小时。

山东大学2010年开发了一款名为SCalf-1的机器人\supercite{27},如图10(c)所示。该机器人共有12个自由度,一个相同的线性液压伺服缸触发所有执行器。该机器人由IC发动机提供动力,可以以1.8m/s的速度以小跑模式运行。2012年,同一个课题组引进了改进型SCalf-2\supercite{28},在液压动力系统、伺服控制器等方面进行了改造,还安装了惯性测量单元、力传感器等。
\begin{figure}[H]
    \centering
    \includegraphics[height=4.5cm]{IMG_10.jpg}
    \caption{国内的机器人}
\end{figure}
\section{现代机器人发展}
2010年,Hutter在苏黎世的瑞士联邦理工学院设计了一只名为Star1ETH的中型犬,如图11所示\supercite{29},该系列高标准弹性执行器安装在star1ETH中,其行为类似于我们的肌腱和肌肉,以暂时保持大量能量。该机器人与IMU结合,IMU提供来自关节​​的运动学信息。另一种多功能四足机器人是为苏黎世 Star1ETH的机器人系统设计的,称为ANYmal,如图11所示\supercite{30}. 它是为石油和天然气平台等具有挑战性的环境中的特殊商业和工业操作而设计的,或者利用其环境感知进行搜索和救援行动。该机器人由高精度执行器驱动,能够实现动态运行步态。小尺寸ANYmal的重量不到 30 公斤,可以携带电池、光学和热像仪、麦克风、动态照明和气体检测传感器等设备。
\begin{figure}[H]
    \centering
    \includegraphics[height=4.5cm]{IMG_12.jpg}
    \caption{star1ETH和ANYmal}
\end{figure}

麻省理工学院于2013年开发了一款名为MIT Cheetah的高效四足机器人\supercite{31}。研究人员实施了四项设计原则,可以减少运动中的能量损失机制。机器人使用的总功率约为973W,运输成本为0.5,与奔跑的动物非常相似。同一研究所在2015年再次开发了MIT Cheetah-2,如图所示。研究人员成功地在机器人中加入了一种新的算法,以稳定的方式实现了从0到4.5 m/s的范围速度的无束缚运行。该机器人可以在跑步机以及草地和不平坦的地形上以受控方式运行,且可以以2.5m/s的速度跳过高达400mm的障碍。
\begin{figure}[H]
    \centering
    \includegraphics[height=4.5cm]{IMG_09.jpg}
    \caption{MIT Cheetah-2}
\end{figure}
\section{总结及未来方向}
可以看出,先前的机器人领域由国外一家独大,而中国并没有相关的研究,但进入21世纪以后,随着中国科技的发展,机器人领域的研究也开始取得很多不俗的成就,现在中国的机器人在世界上是占有一席之地的。无论是在执行器,步态还是性能等方面,中国研制的机器人有了自己的成就。很明显,过去十年将该领域提升到了一个新的水平,同时开辟了新的研究领域,具有新的机遇。我认为未来的机器人应该包括以下内容:

1.高级机器人不仅需要掌握现有对象,还需要解决现实世界环境中的约束,因此应当尝试将机器人与人工智能等领域相融合,不再将机器人只局限于特定领域的开发与应用,应用人工智能或者机器学习相关领域的内容,让机器人获得更强的适应性。

2.需要对轮足混合动力驱动系统进行更多的研究,使其能够结合两种运动形式的优点,如在崎岖地形中移动和在平坦表面上滚动。

3.将机器人应用到更多的领域上,如医药保健领域,商品分销,协同工作等领域。

\newpage
\small

\begin{thebibliography}{99}
    \setlength{\parskip}{0pt}

    \bibitem{1} Meng X, Wang S, Cao Z, Zhang L. A review of quadruped robots and environment perception. In: Control Conference (CCC), 2016 35th Chinese. IEEE; 2016. p. 6350–6356.
    \bibitem{2} Y. Zhong, R. Wang, H. Feng, Y. Chen
    Analysis and research of quadruped robot’s legs: a comprehensive review
    Int J Adv Rob Syst, 16 (3) (2019)
    \bibitem{3} Wikipedia. Mobile Robot. Available: https://en. wikipedia.org/wiki/Mobile robot.
    \bibitem{4} P.G. De Santos, E. Garcia, J. Estremera
    Quadrupedal locomotion: an introduction to the control of four-legged robots
    Springer Science and Business Media (2007)
    \bibitem{5} M.H. Raibert
    Legged robots that balance
    MIT Press (1986)
    \bibitem{6} A.C. Hutchinson
    Machines can walk
    Chartered Mech Eng, 11 (1967), pp. 480-484
    \bibitem{7} P.G. De Santos, E. Garcia, J. Estremera
    Quadrupedal locomotion: an introduction to the control of four-legged robots
    Springer Science and Business Media (2007)
    \bibitem{8} S. Hirose, K. Kato
    April). Study on quadruped walking robot in Tokyo institute of technology
    Proceedings of the 2000 IEEE international conference on robotics and automation (2000), pp. 414-419
    \bibitem{9} M.H. Raibert
    Legged robots that balance
    MIT Press (1986)
    \bibitem{10} S. Hirose, K. Kato
    April). Study on quadruped walking robot in Tokyo institute of technology
    Proceedings of the 2000 IEEE international conference on robotics and automation (2000), pp. 414-419
    \bibitem{11} Hirose S, Yoneda K, Tsukagoshi H. TITAN VII: Quadruped walking and manipulating robot on a steep slope. In: 1997 IEEE international conference on robotics and automation, 1997. Proceedings, vol. 1. IEEE; 1997. p. 494–500.
    \bibitem{12} Arikawa K, Hirose S. Development of quadruped walking robot TITAN-VIII. In: Proceedings of the 1996 IEEE/RSJ international conference on intelligent robots and systems' 96, IROS 96, vol. 1. IEEE; 1996. p. 208–14.
    \bibitem{13} Hodoshima R, Doi T, Fukuda Y, Hirose S, Okamoto T, Mori J. Development of TITAN XI: a quadruped walking robot to work on slopes. In: Proceedings. 2004 IEEE/RSJ International Conference on Intelligent Robots and Systems, 2004. (IROS 2004), vol. 1. IEEE; 2004. p. 792–97.
    \bibitem{14} Komatsu H, Ogata M, Hodoshima R, Endo G, Fukushima EF, Hirose S. Development of quadruped walking robot TITAN XII and its basic consideration on the control of large obstacle traversing motion. Trans JSME (in Japanese) 2014; 80(813).
    \bibitem{15} S. Kitano, S. Hirose, A. Horigome, G. Endo
    TITAN-XIII: sprawling-type quadruped robot with ability of fast and energy-efficient walking
    ROBOMECH J, 3 (1) (2016), p. 8
    \bibitem{16} Buehler M, Battaglia R, Cocosco A, Hawker G, Sarkis J, Yamazaki K. SCOUT: A simple quadruped that walks, climbs, and runs. In: Proceedings 1998 IEEE International Conference on Robotics and Automation, 1998, vol. 2. IEEE; 1998. p. 1707–12.
    \bibitem{17} Battaglia RF. Design of the SCOUT II quadruped with preliminary stair-climbing; 2000.
    \bibitem{18} Ingvast J, Ridderström C, Wikander J. The four-legged robot system WARP1 and its capabilities. In: Second Swedish Workshop on Autonomous Systems; 2002.
    \bibitem{19} P.G. De Santos, E. Garcia, J. Estremera
    Quadrupedal locomotion: an introduction to the control of four-legged robots
    Springer Science and Business Media (2007)
    \bibitem{20} Loc VG, Kang TH, Song HS, Choi HR. Gait planning of quadruped walking and climbing robot in convex corner environment 2005; 2005: 314–9.
    \bibitem{21} Koo IM, Trong TD, Kang TH, Vo G, Song YK, Lee, CM, et al. Control of a quadruped walking robot based on biologically inspired approach. In: IEEE/RSJ International Conference on Intelligent robots and systems, 2007. IROS 2007. IEEE; 2007. p. 2969–74.
    \bibitem{22} I.M. Koo, D.T. Tran, Y.H. Lee, H. Moon, J.C. Koo, S. Park, et al.
    Development of a quadruped walking robot AiDIN-III using biologically inspired kinematic analysis
    Int J Control Autom Syst, 11 (6) (2013), pp. 1276-1289
    \bibitem{23} K. Kotaka, B. Ugurlu, M. Kawanishi, T. Narikiyo
    Prototype development and real-time trot-running implementation of a quadruped robot: RoboCat-1
    2013 IEEE International Conference Mechatronics (ICM), IEEE (2013), pp. 604.-609
    \bibitem{24} M. Li, Z. Jiang, P. Wang, L. Sun, S.S. Ge
    Control of a quadruped robot with bionic springy legs in trotting gait
    J Bionic Eng, 11 (2) (2014), pp. 188-198
    \bibitem{25} F. Gao, C. Qi, Q. Sun, X. Chen, X. Tian
    A quadruped robot with parallel mechanism legs
    2014 IEEE International Conference Robotics and Automation (ICRA), IEEE (2014), p. 2566
    \bibitem{26} G. Junyao, D. Xingguang, H. Qiang, L. Huaxin, X. Zhe, L. Yi, et al.
    The research of hydraulic quadruped bionic robot design
    2013 ICME International Conference on Complex Medical Engineering (CME), IEEE (2013), pp. 620-625
    \bibitem{27} X. Rong, Y. Li, J. Ruan, B. Li
    Design and simulation for a hydraulic actuated quadruped robot
    J Mech Sci Technol, 26 (4) (2012), pp. 1171-1177
    \bibitem{28} X. Rong
    Mechanism design and kinematics analysis of a hydraulically actuated quadruped robot SCalf
    Jinan Shandong University (2013), p. 10
    \bibitem{29} Hutter M, Gehring C, Bloesch M, Hoepflinger M, Siegwart R. Walking and running with StarlETH; 2013.
    \bibitem{30} Robotic Systems Lab, ANYmal. Available: http://www.rsl.ethz.ch/robots-media/animal0.html.
    \bibitem{31} D.J. Hyun, S. Seok, J. Lee, S. Kim
    High speed trot-running: implementation of a hierarchical controller using proprioceptive impedance control on the MIT Cheetah
    Int J Robot Res, 33 (11) (2014), pp. 1417-1445

\end{thebibliography}
\end{document}