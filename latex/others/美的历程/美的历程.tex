\documentclass[12pt,a4paper]{ctexart}
\usepackage{ctex}
\usepackage{emptypage} 
\usepackage{fancyhdr}
\usepackage{amsmath,amsfonts,amssymb,mathtools}
\usepackage{graphicx}
\usepackage{mathptmx}
\usepackage{booktabs}
\usepackage[labelfont=bf]{caption}
\usepackage{indentfirst}
\usepackage{caption}
\usepackage{enumitem}
\usepackage[marginal]{footmisc}
\usepackage{subfigure}
\usepackage{fontspec}
\usepackage{geometry}
\usepackage{setspace}
\usepackage{listings}
\usepackage{xcolor}
\usepackage{float}
\newgeometry{left=3cm,top=2.5cm,bottom=2.5cm,right=3cm}
\setmainfont{Times New Roman}
\setCJKmainfont[BoldFont=SimHei,ItalicFont=KaiTi]{SimSun}

\lstset{
	backgroundcolor=\color{green!10!blue!15},
	rulesepcolor= \color{red!40!blue!100},
	breaklines=true,
	breakatwhitespace=false,
	numbers=left, 
	numberstyle= \small,
	keywordstyle= \color{blue},
	commentstyle=\color{gray}, 
	frame=shadowbox
}

\renewcommand{\baselinestretch}{1.5}

\title{\textbf{对魏晋风度内涵的分析与评价}}

\author{
\\
\Large{麻超 \quad 201300066}
\\[6pt]
{ \large \textit{南京大学人工智能学院}}\\[2pt]\large \textit{maple@smail.nju.edu.cn}
}

\date{}
\newcommand{\supercite}[1]{\textsuperscript{\cite{#1}}}

\begin{document}
\maketitle
\setcounter{page}{1}
\textbf{摘要} \quad {魏晋南北朝是中国历史上一段非常重要的历史,以“乱”著称,在这个时段里,社会剧烈动荡,人民流离失所,同时,传统的两汉经学开始崩塌。在这样的背景下,涌现了这样的一批文人志士,他们感慨社会之不安,对时局不满却又无能为力,转而将感情寄托到对生命的留恋,对田园生活的向往等等。这些行为艺术被后世评价为“魏晋风度”(首见于鲁迅),在嵇康,陶潜之人的文学作品和经历中,我们可以品鉴魏晋风度的形成原因,魏晋风度实际上就是表面行为与内在精神的矛盾,在李泽厚先生《美的历程》一书中,对魏晋风度作了全面的剖析,分析了其形成的原因,指出其矛盾的积极之处。同时当下有一些人仿照魏晋名士,做一些行为艺术,表达对社会的感慨,这也就提醒我们,当下社会需要魏晋风度吗?} \\

\textbf{关键词}\quad {魏晋风度;《美的历程》;矛盾;行为艺术}
\\[60pt]
\section{引言}
在我的六年中学生涯里,我学过很多魏晋文人的作品,现在仍然记得当时在阅读了王羲之“固知一死生为虚诞,齐彭殇为妄作”时内心第一次对生命的意义产生了怀疑,犹记得在学到古诗十九首里“同心而离居,忧伤以终老”时的悲伤心情,以及在学习“五柳先生传”时对陶渊明“不戚戚于贫贱,不汲汲于富贵”的佩服之心……这些活跃在魏晋时期的文人,以他们自己的方式对生命,对理想做出了诠释。他们的作品,他们在作品中所蕴含的精神对我也产生了很大的影响。
\section{魏晋风度的形成因素}
自春秋战国以来,多种思想在战火之中萌发,在那时形成了“百家争鸣”的局面,以孔孟为代表的儒家,以老庄为代表的道家,以商鞅韩非为代表的法家……许多的思想家在这里发表自己的观点。后来,秦统一全国,以法家作为代表思想,汉建立以后,文帝景帝推崇老庄的道家思想,遵循“无为而治”,使百姓得以休息。汉武帝时期,开始实施“罢黜百家,独尊儒术”,自此之后,儒学就成为了整个中国的代表思想。然而经过汉朝四百年的统治之后,战乱不断,不单百姓流离失所,文人士族们也对时局不满。两汉经学在此时开始崩溃,老庄哲学兴起。同时由于曹丕创立了“九品中正制”\supercite{1},导致社会风气浮靡,门阀士族形成,士族弟子终日清淡,不问国事。此时的士族子弟们开始对“人的精神”“文艺作品的内涵”追求更多,认为外在的任何功业事物都是有限和能穷尽的,只是内在的精神本体,才是原始,根本,无限和不可穷尽\supercite{2}。

生于乱世,本就是一种悲哀,在曹魏之后,司马氏统治了整个社会,其高压统治带来了整个社会的不满,八王之乱,永嘉之乱……乱世政治高压,残害异己,平民士人生命朝不保夕,不仅生命安全随时受到威胁,其政治抱负也得不到实现。明末清初思想家王夫之曾经评价道:“孔融死而士气灰,嵇康死而清议绝”,孔融因为热衷于抨击时政,发表激烈言论,与曹操不对付,同时孔融对曹丕娶了甄氏此事也大为不满,所以曹操在公元208年杀害孔融,并株连其全家。司马昭因为嵇康拒不出仕,且为吕安作证而大为不满,故处死嵇康。这二者都是杀鸡儆猴式行为,摆明了告诉全天下的文人,与我作对便是这般下场。前前后后统治者们还杀了不少文人,致使社会动荡。信仰失落的痛苦和官方压抑的恐怖,致使魏晋文人一边精心避祸,一边强行理解,仓猝之间行为乖张,出现了种种独特的风度。

所谓的“魏晋风度”,在我看来,一个很重要的形成因素就是文人士族对传统儒学的一种反抗,这是历史的必然,《古诗十九首》里《驱车上东门》篇便表达了对生命的豁达与对儒学的反抗,“不如饮美酒,被服纨与素”,从这里开始,便展现出了对传统入学的反抗与怀疑,为何怀疑呢?这首诗里也提到了,“人生忽如寄,寿无金石固。万岁更相送,贤圣莫能度。”是啊,人生如此短暂,连圣贤都不能避免最终一死的结局,那么为何不去享受呢?同时,正是因为感受到了生命之短暂,所以这些文人们才开始表现出对生命强烈的留恋之情,将这些写入文章中,对生死存亡的重视,对人生短促的感慨,如王羲之:“古人云:死生亦大矣。岂不痛哉?”,曹操“对酒当歌,人生几何;譬如朝露,去日苦多”。在兰亭集序里,我们能看到的是王羲之对于生命的慨叹,同时达到物我两忘,对潇洒自由的生活的追求不懈\supercite{3}。

除此之外,还有许多的因素与魏晋风度的形成有重要联系,比如社会风气上,由东汉末年的太学“清议”转为“清谈”,由汉末的“月旦”人物转为人物品藻。由品评人物的外貌到品评其内在精神风度。还有文化上,这些文人士族们大多都有深厚的文化素养,所以才能形成一种独特的风度,留下作品供后人品鉴。
\section{魏晋风度的内涵}
那么魏晋风度究竟是什么呢?除了之前提到了对生命的感慨,世人多认为魏晋风度属于怪诞的行为艺术,主要表现在纵酒、服药、品评人物、反抗礼法、追求隐逸生活等等,刘开骅在《魏晋风度:中国文化史上的独特景观》一文中将其概括为: 自信自尊,旷达傲世;蔑视礼法,率性任诞;任性而动,不滞于物;心不染尘,表里澄澈;重情伤情,一往情深;雅量弘度,镇静自若;冶游山水,栖逸林下\supercite{4}。也有一些学者认为其还包括“爱国主义”“乐善行为”“正直精神”等积极行为\supercite{5}。
\section{魏晋风度的代表}
正如李泽厚先生在书中所说,陶潜和阮籍在魏晋时代分别创造了两种迥然不同的艺术境界……他们两个人才是魏晋风度的最优秀代表。从这两个人中间,我们就更能看出魏晋风度的一些更深的东西。阮籍现在的行为艺术在我们看来可能不能理解,然而这些却有着什么的内涵,他就是做给司马昭看的。阮籍在咏怀八十二首里面有两首非常特别,第一首里面写到了:“壮士何慷慨,志欲威八荒。……忠为百世荣,义使令名彰。垂声谢后世,气节故有常。”第二首则有“捐身弃中野,乌鸢作患害。岂若雄杰士,功名从此大。”这两首诗与我们传统看见的“颓废潦倒”的阮籍有着非常大的不同,通篇充满了壮志豪情,然而剩余的八十首却都是哀叹。为何哀叹?因为“时无英雄,竖子成名”,阮籍对这个社会充满了绝望。他目睹了许多残酷的政治清洗和身价毁灭,终年活在忧伤与哀惧之中,欲求解脱而不能,又不适应逆来顺受,故而内心产生了巨大的冲突,借文学作品表达出对当下时局的强烈不满。面对时局,他选择“忧愤无端,慷慨任气”。

陶潜相比之下,就没有这么大的痛苦,然而正如李泽厚先生所说,陶潜实际上也是政治斗争的回避者,陶潜对政治有一定兴趣,然而在见识了社会的残酷之后,他毅然决然地从上层政治中退出,把自己的精神寄托在农村生活上,每日怡然自乐,饮酒种田,发出“富贵非吾愿,帝乡不可期。怀良辰以孤往,或植杖而耘耔。登东皋以舒啸,临清流而赋诗。聊乘化以归尽,乐夫天命复奚疑”的感慨。他在田园生活上真正找到了寄托。面对时局,他走上了不一样的路“超然事外,平淡冲和”。

除此之外,我还想提到嵇康,也就是前文被司马昭所杀的人,他最著名的行为艺术就是在刑场之上,当着所有的太学生弹《广陵散》,感叹到《广陵散》将要失传\supercite{6}。他对司马家的恨是十分深刻的,导致他无法去回避,另一方面来说,他是激进的革命派,号召大家打败司马家,因而遭受迫害。所以,魏晋风度的这些代表人物,都传达了一个核心思想:对时局不满,对统治阶级不满,渴望改变现状却又无能为力。
\section{魏晋风度的评价}
所以可以看出来,魏晋风度实际上是一种矛盾,是属于表面行为与内在精神之间的矛盾,他们生于乱世,他们有抱负,然而时代没有给他们这样的机会,他们只能自己和自己较劲,不得已才有这样的行为。李泽厚先生在书中写道:“在表面看来似乎是如此颓废、悲观、消极的感叹中,深藏着的恰恰是它的反面,是对人生、生命、命运、生活的强烈的欲求和留恋。”魏晋风度颓废与热爱并存,正因为热爱生命,正因为想把生命活得更有意义,所以才与那时局,那命运对抗。同时,也正是因为对时局的无能为力,所以才会有忽视外在的虚幻,而去追求内心的愉悦、进而活出了自我,做到了“知行合一”。
\section{我们需要魏晋风度吗}
坦白来说,这个问题是十分尖锐的。正如前文总结的魏晋风度的内涵,它形成的原因是对时局的不满,以及自身抱负无法实现。事实上,从古至今的每一代人都有一个错觉,那就是之前的时代更好,更有发展前途,比如说现在会有人羡慕改革开放之前的时代,以及我们经常能在影视剧里看到民国的一些百姓说着要是大清还在就好了。究其原因,是因为眼前的问题被不断放大,导致看不见当下社会的红利,而去羡慕之前一个时代的某些制度,进而羡慕那个时代。比如说当下有些人选择去追求改革开放前国内相对公平的分配制度,这当然没有错,然而如果去羡慕那个时代就有问题了,他们忽视了社会生产力的巨大发展,只因对当下不满所以批判一切新的,怀念一切旧的事物。

其实,很多人所羡慕的那些“前尘往事”,本质上还是存在于社会精英阶层,只是平凡无味的生活中浅浅的一抹亮色,与大多数普通人都无关。千年前与千年后,并没有太大区别,只是文明发展给普通人带来了更多的便利。社会对于普通大众是残酷的,每个时代都是这样,我们今日看到百姓的权益得不到声张,却也应该想到,数百年前,百姓还只是统治阶级眼中的蝼蚁,随意蹂躏。八十年前,蒋介石为了阻止日军进攻炸开了花园口,三百万百姓死亡,数千万群众流离失所,那个时候的他们或许也只想要一个安分日子,而今日我们,还愿意去回到那里吗?始终记得,我们只是平民百姓,以那些精英阶级的生活代替整个社会的面貌是不合理且不现实的。不用去羡慕别人,或许百年后,也会有一群这样的郁郁不得志的人,在回首往事时想,如果我能回到百年以前就好了。每个时代“既是最坏的时代,也是最好的时代”。

魏晋风度涌现出了一大批的文人志士,然而当我们回顾他们的精神的时候会发现,他们的矛盾来自于对时局的不满,是对司马家族高压统治的反抗,所以坦白来说,他们是好人,是有勇气有良心的中国人,对社会抱有极大的同情心,渴望社会变得更好,然而却因为种种阻碍只得自己唉声叹气或者寄身田园之间。对于我们来说,这种忧国忧民之心是非常值得学习的,然而要去行动,要去努力实现自己的抱负。莫泣穷途老泪,休怜儿女新亭!
\small

\begin{thebibliography}{99}
    \setlength{\parskip}{0pt}

    \bibitem{1} Wikipedia.九品中正制 https://zh.m.wikipedia.org/zh-cn/九品中正制
    \bibitem{2} 李泽厚.美的历程 p. 130
    \bibitem{3} 刘庆华.从《金谷诗序》《兰亭集序》看两晋文人的生存选择与文学选择[J].广州大学学报:社会科学版,2006(3):91-96.
    \bibitem{4} 南京政治学院学报,2003(04):75-79
    \bibitem{5} 孙秀彬,赵百成《〈世说新语〉中「魏晋风度」浅说》(佳木斯师专学报,1996(03):53-57)
    \bibitem{6} 《世说新语笺疏·雅量第六》引《文士传》:康颜色不变,问其兄曰:“向以琴来不邪?”兄曰:“以来。”康取调之,为太平引,曲成,叹曰:“太平引于今绝也!”
\end{thebibliography}
\end{document}