\documentclass[12pt,a4paper]{ctexart}
\usepackage{ctex}
\usepackage{emptypage} 
\usepackage{fancyhdr}
\usepackage{amsmath,amsfonts,amssymb,mathtools}
\usepackage{graphicx}
\usepackage{mathptmx}
\usepackage{booktabs}
\usepackage[labelfont=bf]{caption}
\usepackage{indentfirst}
\usepackage{caption}
\usepackage{enumitem}
\usepackage[marginal]{footmisc}
\usepackage{subfigure}
\usepackage{fontspec}
\usepackage{geometry}
\usepackage{setspace}
\usepackage{listings}
\usepackage{xcolor}
\usepackage{float}
\newgeometry{left=3cm,top=2.5cm,bottom=2.5cm,right=3cm}
\setmainfont{Times New Roman}
\setCJKmainfont[BoldFont=SimHei,ItalicFont=KaiTi]{SimSun}

\lstset{
	backgroundcolor=\color{green!10!blue!15},
	rulesepcolor= \color{red!40!blue!100},
	breaklines=true,
	breakatwhitespace=false,
	numbers=left, 
	numberstyle= \small,
	keywordstyle= \color{blue},
	commentstyle=\color{gray}, 
	frame=shadowbox
}
\CTEXsetup[format={\Large\bfseries}]{section}
\renewcommand{\baselinestretch}{1.5}

\title{\textbf{浅谈逻辑学}}

\author{
\\
\Large{麻超 \quad 201300066}
\\[6pt]
{ \large \textit{南京大学人工智能学院}}\\[2pt]
}

\date{}
\newcommand{\supercite}[1]{\textsuperscript{\cite{#1}}}

\begin{document}
\maketitle
\setcounter{page}{1}
这个学期我有幸接触到一门神奇且睿智的学科——逻辑学。事实上,我们每个人在很早的时候就有了逻辑的意识,平时也总是要求讲逻辑,但并不深入。
其实生活处处存在逻辑学,它融入在我们生活的每个角落里,在每一次交谈的语句里,在每一次落笔的措辞里,在每一次思考的环节里。我们与逻辑学频繁接触,只是并未认真察觉。

培根曾言:“读史使人明智,读诗使人灵秀,数学使人严密,物理学使人深刻,伦理学使人庄重,逻辑学、修辞学使人善辨,凡有学者,皆成性格”。逻辑学就是这样一门严密却又神奇的学科,无论是平时说话做事,还是搞学术研究都离不开逻辑学的支持。逻辑之于生活,就像是水之于生命,饭菜之于盐,生活中没有逻辑就相当于生命没有活动的规则和定律而一塌糊涂,就像是食之无味的饭菜吊不起胃口。

不管我们是否在意,逻辑在生活中都会被经常用到,通过逻辑学习,我们便可以更加准确更加灵活的运用逻辑,让生活更有规律,让言语更加活泼和不至于犯基本的逻辑错误让人耻笑。逻辑有点像随处可见的水,不显眼,很容易被忽略,但人人都离不开它。我们说话的时候、考试的时候,其实都用到了它,如果我们说话不讲逻辑就会意思表达不清楚造成歧义,“逻辑是生活中寻求满足其愿望的实际工具”。它为我们提供了更多的便利。逻辑在学习中处处用到,比如说最现实的,要写一篇文章,你就要好好地组织这篇文章,如果是议论文,那就要更加重视这篇文章的构
造和逻辑结构,什么在先什么在后,什么应该先说,什么应该放到,后面说,一篇组织好的文章,也就是有内在逻辑的文章让人一看就懂,一眼就能明白整篇文章到底写
了什么,要表达什么意思。而一篇逻辑论乱的文章,则会让人感到很烦躁,甚至读都不想读,就扔在一边了,这就是问什么文章要讲究结构的原因,究竟是总分总还是其
他的模式,或者是散文重要的有一个内在的逻辑在里面。

上个学期,我院开设了一门“数理逻辑”的课程,这正是逻辑学在理科学术研究上的应用,在数理逻辑这门课里,我们学习了命题逻辑,一阶逻辑的语言、自然推理系统、永真推理系统,完全性定理等内容,逻辑思维与导论这门课里提到的蕴含,命题等内容,在这门课里都有深入的介绍,当时学习这门课时,只觉得晦涩难懂,各种各样的定义与证明,但是现在看来,全书充满了严密的逻辑,为在高中接触过简单逻辑学的我们指了一条更远的路,人工智能的研究离不开数理逻辑的协助,在自然语言处理等方面会广泛应用数理逻辑的知识。

我曾看过几场辩论赛,无论是正规的辩论形式还是像《奇葩说》这样的节目,都充满了严密的逻辑,每位辩手都在场上认真地寻找对方的失误,寻找可能存在的切入点,每位选手都用他们严密的逻辑思维为我们带来尽可能完美的表现。我时常惊叹于他们的思想,可以让别人的思维随着自己变化,用一些并不晦涩的话术,为众人带来一场饕餮大餐。看的越多,越感觉到自己在逻辑思维方面的欠缺,所以自己在逻辑思维能力方面要不断提高,我会努力,在生活中要不断学习,汲取我不断向前的养分。课堂上的老师讲的一些逻辑小故事更是引起我们逻辑思考的积极性。当我们阅读福尔摩斯的时候我们总是情不自禁的被吸引下去,这就是逻辑的魅力,有了逻辑我们的生活便会减少很多的欺骗和不理智买我们会生活在一个理智的空间内,不至于由于自己的冲动做错事从而导致无法挽回的后果。换句话说,有了逻辑生活就井然有序,不再混沌不堪;有了逻辑生活就会显得阳光灿烂,不再一片愁云;有了逻辑生活就会充满一直向前的动力。

有句古语是“一朝被蛇咬,十年怕井绳”,这种假想的思维逻辑乍一看好像没什么,但是会对自己产生巨大的困扰。有一个真实的例子,一个女生,在初一的时候和好朋友外出旅行。等车的时候,她的好朋友买零食回来朝她打招呼,但此时一辆卡车从后面飞快地冲过来,转眼间,好朋友就被压在了车轮下,当场死亡。亲眼看到好朋友的死亡过程,她的心灵中种下了阴影。在此之后,她总是想自己也会那样死掉。这种假想的思维逻辑,让她变得非常恐慌,不敢独自外出行动。从此以后,她害怕所有人,害怕自己的父母,甚至害怕在镜子里的自己,把自己关在房间里,连听到汽车的声音都会发抖。现在的她已经完全被恐惧所控制,这是她在思维中用逻辑思考死亡的结局,这种困惑,伤害的是她自己,毁灭的是她自己的前途。现在她唯一要做的就是敢于超越逻辑思维,突破幻想,从而战胜自己的恐惧。

我们还看到,在当今的社会生活中,逻辑缺失和混乱现象十分严重。从人们日常的语言交流,到明星访谈、官员讲话;从广泛的传媒报道,到图书论文、法律条规。几乎时时处处都能看到概念不明确,判断不准确,推理不正确,论证不科学,自相矛盾,前后冲突,甚至整个思维过程混乱不堪,让人不知所云的现象存在。这些逻辑问题妨碍着人们正常的社会生活,有时甚至造成十分严重的后果。因此,学好逻辑学真的很重要,不仅对个人,更是对整个社会。大千世界虽然纷繁复杂,但只要有心,我们就能挖掘生活中的逻辑问题。

作为新一代大学生的我们,懂得运用所学的逻辑知识去看待问题,更应该懂得在实际生活中如何运用逻辑去解决问题,优化自己的逻辑思维能力,让自己但生活充满阳光,自信的对待生活中的每一个难题。
\end{document}