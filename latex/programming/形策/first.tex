\documentclass[12pt,a4paper]{article}
\usepackage{ctex}
\usepackage{emptypage} 
\usepackage{fancyhdr}
\usepackage{amsmath,amsfonts,amssymb,mathtools}
\usepackage{graphicx}
\usepackage{mathptmx}
\usepackage{booktabs}
\usepackage[labelfont=bf]{caption}
\usepackage{indentfirst}
\usepackage{caption}
\usepackage{enumitem}
\usepackage[marginal]{footmisc}
\usepackage{subfigure}
\usepackage{fontspec}
\usepackage{geometry}
\usepackage{setspace}
\usepackage{listings}
\usepackage{xcolor}
\usepackage{float}
\newgeometry{left=3cm,top=2.5cm,bottom=2.5cm,right=3cm}
\setmainfont{Times New Roman}
\setCJKmainfont[BoldFont=SimHei,ItalicFont=KaiTi]{SimSun}

\lstset{
	backgroundcolor=\color{green!10!blue!15},%代码块背景色
	rulesepcolor= \color{red!40!blue!100}, %代码块边框颜色
	breaklines=true,  %代码过长则换行
	breakatwhitespace=false,
	numbers=left, %行号在左侧显示
	numberstyle= \small,%行号字体
	keywordstyle= \color{blue},%关键字颜色
	commentstyle=\color{gray}, %注释颜色
	frame=shadowbox%用方框框住代码块
}

\renewcommand{\baselinestretch}{1.5}%可加可不加,控制行间距

\title{\textbf{形势与政策第一次作业}}%标题

\author{
\\
\Large{麻超 \quad 201300066}
\\[6pt]
{ \large \textit{南京大学人工智能学院}}\\[2pt]
}

\date{}
\newcommand{\supercite}[1]{\textsuperscript{\cite{#1}}}

\begin{document}
\maketitle
\setcounter{page}{1}
转眼间一年就过去了,今日是2021年9月5日,我们是在2020年9月6日来到学校的,恰好一年时间,我的大一生活也画上了一个句号。回望一年,初来南大的我显得十分迷茫,一直不断尝试却没有好的结果。我的大一过得并不如意,因为脱离了高中时的奋发上进的状态,来到自由的大学,显然还没有找到属于自己的道路,希望能够勇敢探索,走入正轨。

大一的这一年是尝试的一年,可对我来说尝试到最后选择的更多是逃避。在学习方面,由于我来自甘肃省,相较之下学习基础较为薄弱,由于时间安排不合理,沉迷享乐等因素,只能说成绩平平。大一上学期的成绩就比较一般,学分绩为4.13左右,排名在七十多。大一上学期也听取了很多的报告,却没有起到好的激励效果,在第二学期显得更加闲散,甚至数学分析(二)也只有54分,免不了要补考,包括其他的几门学科,如高等代数(二),还有数电,都是60分出头,都属于老师的努力。由于大一这一年没有主动学习过英语,靠着高三的老底,英语四级刚好过了分数线。所以对我而言,大一的学习是做的非常不好的。在其他方面,大一初我当选了AI3班的团组织委员,这一年工作还是比较负责,和团支书搭配得也比较好。在学生会方面,我成为了人工智能学院宣传部的一名部员,这一年在习三卓,高辰潇等学长的带领下学习了不少技能,也有了一些满意的作品,在大一下学期快结束的时候,当选了新一届宣传部的副部长。社团活动方面,大一的时候加入了南京大学天文爱好者协会,参加了不少的社团活动,拓展了视野,也收获了许多朋友,总的来说还是不错。

总而概之,大一的学习生活比较失败的原因还是在于懈怠与闲散,没有提起紧张意识,总是没有好好学习的意识,所以才造就了今天的局面。目前我相对于其他同学已经差下许多,要在大二树立起良好的意识,认真学习,弥补先前犯下的过错,过一个完整并成功的大学生活。

大二这一年要学习的内容是很多的,难度也是上了一个档次,我自然要做好完整的规划,认真完成作业,学习技能,不可以再掉以轻心。在其他方面,也要做到认真负责。

关于大二的三个目标,第一个目标自然是认真完成学业,做到不懈怠,不迟到不早退,认真完成每次作业,考试也要认真复习,尽量做到不掉队,切实学到有用的知识和技能。第二个是认真完成宣传部副部长的工作,从新一年学生会的招新工作开始,认真对这一年的宣传部工作,一方面提高自身在宣传工作里的水平,做出更多好的成果,另一方面也要带领新的宣传部成员,共同进步。第三个目标是提高自身的社交水平,比如更多地参加社会实践、志愿活动、社团活动等,不再沉迷于自己的小世界里,走出去寻求更大的世界,谋求进步。

大二的这一年是充满挑战的一年,也是充满机遇的一年,因为大一已经走过迷惘,过错的一段时间,大二更要花时间去弥补,期待大二能够更加成功,遇见一个更好的自己。


\end{document}