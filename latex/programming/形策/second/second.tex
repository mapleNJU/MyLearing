\documentclass[12pt,a4paper]{article}
\usepackage{ctex}
\usepackage{emptypage} 
\usepackage{fancyhdr}
\usepackage{amsmath,amsfonts,amssymb,mathtools}
\usepackage{graphicx}
\usepackage{mathptmx}
\usepackage{booktabs}
\usepackage[labelfont=bf]{caption}
\usepackage{indentfirst}
\usepackage{caption}
\usepackage{enumitem}
\usepackage[marginal]{footmisc}
\usepackage{subfigure}
\usepackage{fontspec}
\usepackage{geometry}
\usepackage{setspace}
\usepackage{listings}
\usepackage{xcolor}
\usepackage{float}
\newgeometry{left=3cm,top=2.5cm,bottom=2.5cm,right=3cm}
\setmainfont{Times New Roman}
\setCJKmainfont[BoldFont=SimHei,ItalicFont=KaiTi]{SimSun}

\lstset{
	backgroundcolor=\color{green!10!blue!15},%代码块背景色
	rulesepcolor= \color{red!40!blue!100}, %代码块边框颜色
	breaklines=true,  %代码过长则换行
	breakatwhitespace=false,
	numbers=left, %行号在左侧显示
	numberstyle= \small,%行号字体
	keywordstyle= \color{blue},%关键字颜色
	commentstyle=\color{gray}, %注释颜色
	frame=shadowbox%用方框框住代码块
}

\renewcommand{\baselinestretch}{1.5}%可加可不加,控制行间距

\title{\textbf{形势与政策第二次作业}}%标题

\author{
\\
\Large{麻超 \quad 201300066}
\\[6pt]
{ \large \textit{南京大学人工智能学院}}\\[2pt]
}
\date{\today}

\newcommand{\supercite}[1]{\textsuperscript{\cite{#1}}}

\begin{document}
\maketitle
\setcounter{page}{1}
\section{如何理解我国发展仍处于重要战略机遇期}
随着十三五计划的顺利完成,我国在中华民族伟大复兴的道路上迈出了新的一大步,正如习总书记所说,我国目前正处于百年未有之大变局,一方面,新冠疫情形势愈发严峻,而我国目前正可以趁着这大好局势发展,另一方面,以美国为首的国家对华政策愈发收紧,外部风险加大。在这样的形势下,我国正面临着机遇和挑战,但是机遇大于挑战。
\par
第一方面,自改革开放以来,我国的经济实现了巨大腾飞,创造了发展的奇迹,2020年我国国民生产总值已经超过100万亿元,稳坐全球经济第二把交椅。另一方面,我国的经济增长也保持着较高速率,目前在每年6\%左右,因为新冠疫情的原因,我国在全球主要经济体国家中经济韧性较好,拉动全球经济发展。第二方面,在2020年,我国实现了脱贫攻坚的重要任务,截止目前,我国只有约500万人口活在贫困线以下,实现了前所未有的奇迹。
\par
另一方面,国际形势的变化对我国来说既是挑战又是机遇。自2001年911事件以来,美国在中东战场上深陷泥潭,直至一个月前才从阿富汗撤军,而中国就抓住了这20年时间,疯狂发展。自2016年以来,美国将矛头又对准了中国,但是一方面,由于特朗普在政治上的失势,将中国和俄罗斯绑在了同一条战船上,同时,由于特朗普在任期间废除了TPP等协议,许多美国的盟友都动了其他心思,比如前日法国召回驻美国和澳大利亚大使。在前几年的经济斗争里,由于中华民族的和平等优秀品质,使中国的力量得以集中,在贸易战中没有受到太大的损失。2020年,特朗普下台,美国陷入大乱,新冠疫情形势愈发严峻,美股数次熔断,美国在国际上的公信力得到了很大的削减,而在这时,中国上下一心,集体发展经济,突破制裁,如今的中国,在往常我们不擅长的领域都在集中力量办大事,取得了很大的突破。
\par
由新冠疫情衍生出的机遇期还会有很长一段时间。中美博弈也进入了白热化阶段,在现在的形势下,中国人民齐心协力,定能从此突破美国的制约,取得极大战略主动权。
\section{为什么说“集中精力办好自己的事情,这是我们赢得主动、赢得优势、赢得未来的关键”}
中华民族自数千年来,就秉持“自我为主,独立自主,自力更生,自强不息”的品质,这是中华文明政治哲学的精华。我们是一个和平的民族,秉持着和平的理念,我们在数千年来,赢得了许多朋友,新中国成立以后,周恩来总理在1953年接见印度政府代表团时,提出了和平共处五项原则,即:互相尊重领土主权(在亚非会议上改为互相尊重主权和领土完整)、互不侵犯、互不干涉内政、平等互惠(在中印、中缅联合声明中改为平等互利)和和平共处。习近平总书记在数次讲话中也提到要坚持和平发展的理念,共同构建人类命运共同体。由此可见,坚持和平是中华民族的共识。
\par
另一方面,美国自二战以来,尤其是苏联解体以后,数次以美国霸权为中心,不断打压周边国家,屡次发动军事侵略,欧盟国家助纣为虐,在这几十年里,败坏自己的好感,众多国家对美国为首的霸权主义国家已是忍无可忍。全球人民是一个共同体,只有坚持和平的理念才能够赢得好感,实现中华民族伟大复兴。

集中力量办好自己的事情,是只有社会主义国家才能够完成的事情,在旧社会里,阶级矛盾对立,人民自己尚不能够吃饱穿暖,何来的力气为政府办大事?在资本主义社会里,更是人人为利,以民国为例,蒋介石的国民政府四大家族垄断了几乎所有产业,不为民众考虑,将经济搞得一塌糊涂,最终在人民群众的洋流中倒台,更不必谈集中力量办大事。所以,几乎只有社会主义国家才能够完成集中精力办好大事。

集中力量办好自己的事情,才能在当今国际博弈中有更强的力量,有更深厚的底蕴,在国与国博弈中占据一份主动。当下,我认为我们应该集中力量办的大事就是发展经济,发展科技,解放台湾。发展经济和科技一方面是为了在当下中美博弈的局势下,得到更多的力量,另一方面就是切实提高人民群众的生活发展水平。解放台湾是中美对抗的副产物。台湾自古以来是中国不可分割的一部分,近年来,台独势力在台湾十分猖獗,为了切实维护中华民族的领土完整,实现中华民族统一大业,解放台湾是必不可少的,是重中之重。

集中力量办好大事,是让我们在对抗中实现主动,让我们赢得优势,赢得未来必不可少的条件。

\end{document}